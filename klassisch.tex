\section{Klassische Erkl\"arungen}\label{klassisch}

Bevor man neue Theorien aufstellt um diese ungew\"ohnliche
Beschleunigung zu erkl\"aren, sollte man sich zun\"achst die
klassischen Erkl\"arungen ansehen, die die Beschleunigung erkl\"aren
k\"onnten bzw. eine Fehlerabsch\"atzung der Beschleunigung geben
k\"onnen. Hierzu teilte Anderson et al. \cite{Anderson2002} die Daten von Pioneer 10 in drei Intervalle
ein, die man durch den steigenden Abstand zur Sonne und Einfl\"usse von
anderen Himmelsk\"orpern abgrenzen muss. (Abb. \ref{fig:intervalle})

\begin{figure}[htbn]
\begin{center}
\noindent    
\psfig{figure=images/intervalle,width=0.8\linewidth,height=\textheight,keepaspectratio}
\end{center}
\vskip -10pt
  \caption{Abgrenzung der drei Datenintervalle von Pioneer 10 \cite{Anderson2002}}
\label{fig:intervalle}
\end{figure} 

\subsection{Externe Fehlerquellen}\label{extern}

\subsubsection{Strahlungsdruck der Sonne}

Durch den Impuls den die Photonen des Sonnenlichts auf die Fl\"ache der Sonde
\"ubertragen entsteht eine Kraft und damit letztlich auch eine
Beschleunigung. Diese ist zwar in Flugrichtung gerichtet, muss aber
f\"ur eine Fehlerrechnung und einer genauen zuerst betrachtet werden. Ein Modell f\"ur
den Strahlungsdrucks gab es schon vor Pioneer 10 und 11. Dieses Modell
formuliert eine Beschleunigung in Abh\"angigkeit der Ausrichtung und
Entfernung der Sonde zur Sonne.

\begin{equation}\label{eins}
a_{\mathit{sd}}(r)=-\frac{\kappa f_{s}A\cos \theta
(r)}{cMr^{2}}
\end{equation}

Dabei sind:
\begin{itemize}
\item $\mathit{f_s}=1367W/(\mathit{m\cdot AU})^2$
\item $A=r^2\pi =(1,37m)^2\pi $ (als sonnenzugewandte Oberfl\"ache der Sonde wird vereinfacht
die Fl\"ache der Parabolantenne verwendet)
\item $\theta $ ist der Winkel unter dem die Photonen auf $A$ treffen. Da der Winkel immer unter 1.5° ist wird er zu $\theta = 0\degree$ vereinfacht, was lediglich zu einem Verlust an Genauigkeit von $< 4 \cdot 10^{-12} \frac{cm}{s^2}$\cite{Markwardt2002} führt.
\item $c$ ist die Lichtgeschwindigkeit
\item $M$ ist die nominelle Masse der Sonde, zum Zeitpunkt an welchem die H\"alfte des
Treibstoffs verbraucht war (F\"ur Pioneer 10 wurden 241 kg angenommen)
\item $r$ ist der Abstand Sonde-Sonne in AU
\item $\kappa $ ist der effektive Absorptions/Reflexions-Koeffizient. F\"ur
Pioneer 10 wurde ein $\kappa_0=1,71 $ ermittelt\cite{Anderson2002}.
\end{itemize}

\bigskip

So erh\"alt man bei einem Abstand von 5,2 AU eine Beschleunigung von
$-(70,0\pm 3,5)\cdot 10^{-8}\mathit{cm}/s^{2}$ mit $\kappa_{5,2 AU}=1,77$.
Da der Strahlungsdruck wie alle anderen
Gesetze im 3-dimensionalen Raum dem $1/r^{2}$ Gesetzt folgt, war die
Beschleunigung bei 10 AU noch $-18,9\cdot 10^{-8}\mathit{cm}/s^{2}$
und bei 70 AU nur noch $-0,39\cdot 10^{-8}\mathit{cm}/s^{2}$. Da der
Strahlungsdruck der Sonne sehr genau von den Auswertungsprogrammen
modelliert werden kann, liegt der Fehler f\"ur Pioneer 10 zwischen 40
und 70 AU bei $0,001\cdot 10^{-8}\mathit{cm}/s^{2}$ und f\"ur Pioneer
11 zwischen 22 und 32 AU bei $0,006\cdot 10^{-8}\mathit{cm}/s^{2}$

Da die Masse durch die Verbrennung von Treibstoff mit der Zeit auch
ver\"anderte, erh\"alt man f\"ur die Ver\"anderung von $a_{p}$ f\"ur
die 3 Intervalle\cite{Anderson2002}:

\begin{equation}
\mathit{\delta a}_{p}=[(0,040\pm 0,035),(0,029\pm 0,025),(0,020\pm
0,017)]\cdot 10^{-8}\mathit{cm}/s^{2}
\end{equation}
Man bildet das gewichtete Mittel und erh\"alt f\"ur

\begin{equation}
a_{\mathit{sd}}=(0,03\pm 0,01)\cdot 10^{-8}\mathit{cm}/s^{2}
\end{equation}

Wenn man nun f\"ur Pioneer 11 das gleiche durchf\"uhrt erh\"alt man
\begin{equation}
a_{\mathit{sd}}=(0,09\pm 0,21)\cdot 10^{-8}\mathit{cm}/s^{2}
\end{equation}

Dies kann nicht die Anomalie erkl\"aren. Ist jedoch f\"ur die
Fehlerrechnung interessant.


\bigskip

\subsubsection{Der Sonnenwind}

Der Sonnenwind beschleunigt die Pioneers \"ahnlich wie Gl. \ref{eins}, nur das
man $\frac{f_{s}}{c}$ \ durch $m_{p}v^{2}n$ ersetzt. Hier ist
$n{\approx}5/\mathit{cm}^{3}$ die Protonendichte bei 1 AU und
$v{\approx}400\mathit{km}/s$ die Windgeschwindigkeit. So erh\"alt
man\cite{Anderson2002}:
\begin{equation}
a_{\mathit{sw}}(r)=-\frac{\kappa _{\mathit{sw}}m_{p}v^{2}nA\cos \theta
}{Mr^{2}}\approx -1,11\cdot
10^{-11}(20\frac{\mathit{AU}}{r})^{2}\mathit{cm}/s^{2}
\end{equation}

Da die Protonendichte um 100\% schwanken kann, ist die tats\"achliche
Beschleunigung unvorhersehbar. Unter der konservativen
Annahme, dass diese Beschleunigung um zwei Gr\"o{\ss}enordnungen
kleiner ist als die des Strahlungsdrucks, ist sie zu vernachl\"assigenden.


\bigskip

\subsubsection{Die Effekte der Sonnencorona}

Wie in Abschnitt 3.1.2 gesehen, ist der Effekt des Sonnenwindes auf die
Beschleunigung der Sonden vernachl\"assigbar. Jedoch muss man den
Einfluss der Sonnencorona auf die Radiosignale ber\"ucksichtigen. Denn
die Elektronendichte und der Gradient der Elektronendichte beeinflussen
die Ausbreitung von Radiowellen in einem Medium. Die Zeitverz\"ogerung einer S-Band Welle auf einem Weg $l$ l\"asst sich beschreiben als
\begin{equation}
\Delta t=\frac{\pm 1}{2\mathit{cn}_{\mathit{krit}}(\nu )}\int
_{\mathit{Sonnenmittelpunkt}}^{\mathit{Sonnencorona}}n_{e}(t,r)\mathit{dl}
\end{equation}
\cite{Anderson2002}

Mit
\begin{itemize}
\item $n_{\mathit{krit}}(\nu )=1,240\cdot 10^{4}(\frac{\nu
}{1}\mathit{MHz})^{2}\frac{1}{\mathit{cm}^{3}}$ ist die kritische
Plasmadichte f\"ur eine Tr\"agerfrequenz $\nu $

\item $n_{e}(t,r)$ ist die freie Elektronendichte im Sonnenplasma
\item Das positive Vorzeichen ist f\"ur Laufzeitdaten und das negative
f\"ur Dopplerdaten
\end{itemize}

Um den Einfluss der Sonnencorona auf die Radiowellen der
Pioneer-Sonden zu verstehen, kann die Elektronendichte der Corona als
Summe einer statischen und einer ver\"anderlichen Elektronendichte
modelliert werden. Da der ver\"anderliche Teil \ kaum Einfluss auf
Dopplerdaten hat\cite{Anderson2002}, reicht es ein statisches Modell der Sonnencorona
zu betrachten. F\"ur dieses Modell erh\"alt man eine freie
Elektronendichte von\cite{Anderson2002}:

\begin{equation}
n_{e}(r,t)=A(\frac{R_{s}}{r})^{2}+B(\frac{R_{s}}{r})^{2,7}\cdot
e^{-[\frac{\Phi }{\Phi _{0}}]^{2}}+C(\frac{R_{s}}{r})^{6}
\end{equation}

Aus den Daten der Cassini-Mission wurden f\"ur die Parameter A,B und C die
folgenden Werte ermittelt:

\begin{equation*}
A=6,0\cdot 10^{3}m
\end{equation*}
\begin{equation*}
B=2,0\cdot 10^{4}m
\end{equation*}
\begin{equation*}
C=0,6\cdot 10^{6}m
\end{equation*}
Dies nennt man das {\quotedblbase}Cassini Corona Model``. Die
Auswertungsprogramme ODP/Sigma und CHASMP haben f\"ur den Fehler der Beschleunigung aufgrund der Sonnencorona den
Wert

\begin{equation}
\sigma _{\mathit{corona}}=\pm 0,02\cdot
10^{-8}\mathit{cm}/s^{2}
\end{equation}
berechnet\cite{Anderson2002}.


\bigskip

\subsubsection{Lorentzkr\"afte}

Es ist nicht unwahrscheinlich dass die Sonden eine Ladung tragen, die im
elektromagnetischen Feld des Sonnensystems einen Einfluss auf die
Geschwindigkeit hat. Die magnetische Feldst\"arke im \"au{\ss}eren
Sonnensystem liegt bei unter $10^{-5}\mathit{Gauss}$\cite{Anderson2002}.
Unter der Annahme dass die Sonden eine maximale Ladung von $0,1-1,8\mu
C$\footnote{ %% ### Fussnote nicht mehr auf zwei Seiten splitten
Ich gebe hier deshalb einen Bereich an, da die Angaben in \cite{Null1976} auf der Tatsache beruhen, dass
Pioneer 10 bei Jupiter eine magnetische Feldst\"arke von 1,135 Gau{\ss}
gemessen h\"atte. Dies ist aber laut \cite{Anderson2002}, S. 29 falsch, da f\"ur Pioneer 10 nur eine
magnetische Feldst\"arke von 0,185 Gau{\ss} gemessen wurde (Pioneer 11
ma{\ss} durch ihre gr\"o{\ss}ere Ann\"aherung 1,135 Gau{\ss}).
\begin{equation*}
B=\frac{F_{B}}{\mathit{qv}}\ \ \mathit{mit}\ \ F=m\cdot
a\ \ \mathit{ergibt}\mathit{sich}\mathit{f\text{\"u}r}\mathit{die}\mathit{Ladung}\ \ q=\frac{\mathit{ma}}{\mathit{BV}}
\end{equation*}
mit \cite{Anderson2002}, S. 29:
\begin{equation*}
\begin{gathered}v=14,36\cdot 10^{3}m/s;\ B=0,185\cdot
10^{-4}T;\ m=241\mathit{kg};\ a=20\cdot 10^{-10}m/s^{2}\end{gathered}
\end{equation*}
ergibt sich:
\begin{equation*}
q=1,8143\mu C
\end{equation*}}
tragen k\"onnen errechnet sich eine Beschleunigung von
\begin{equation}
a=\frac{\mathit{Bqv}}{M}=\frac{1\cdot 10^{-9}T\cdot 1\cdot
10^{-6}C\cdot 14,36\cdot 10^{3}m/s}{251,883\mathit{kg}}=5,7\cdot
10^{-14}m/s^{2}
\end{equation}
Mit 
\begin{itemize}
\item $B$ ist die magnetische Flussdichte
\item $q$ ist die Ladung der Sonde
\item $v$ ist die Geschwindikeit der Sonde
\end{itemize}

Diese Beschleunigung kann man vollst\"andig vernachl\"assigen.


\bigskip

\subsubsection{ Die Gravitation des Kuiperg\"urtel}

Unter der Annahme, dass sich im Kuiperg\"urtel etwa 1 Erdmasse an
Partikeln befinden hat man 3 Staubverteilungen gepr\"uft: 1) eine
gleichm\"a{\ss}ige Verteilung, 2) eine 2:1 Resonanz Verteilung \ mit
einem Maximum bei 47,8 AU und 3) eine 3:2 Resonanz Verteilung mit einem
Peak bei 39,4 AU. Die letzten 2 Verteilungen wurden deswegen
ausgew\"ahlt, da diese Verh\"altnisse bei dem Resonanzeffekt von Neptun
auf Pluto beobachtet wurden. Abb.\ref{fig:kuiper} zeigt die Beschleunigung auf
Pioneer 10 von 30 bis 65 AU. Hier erkennt man eine Beschleunigung in der
Gr\"o{\ss}enordnung von $10^{-9}\mathit{cm}/s^{2}$. \ In Abb.\ref{fig:kuiper}
erkennt man zus\"atzlich, dass die Beschleunigung nicht konstant ist.
Da der Wert zwei Gr\"o{\ss}enordnungen unter der Anomalie liegt und
nicht konstant ist, kann man den Kuiperg\"urtel als Ursache der
Anomalie ausschlie{\ss}en. Infrarotmessungen von 2002 haben eine Masse
von ca. 0,3 Erdmassen im Kuiperg\"urtel entdeckt. Dies wird in der
Fehlerrechnung mit \ $\sigma _{\mathit{KG}}=\pm 3\cdot
10^{-10}\mathit{cm}/s^{2}$ ber\"ucksichtigt\cite{Anderson2002}.


\begin{figure}[htbn]
\begin{center}
\noindent    
\psfig{figure=images/kuiper,width=0.5\linewidth,height=\textheight,keepaspectratio}
\end{center}
\vskip -10pt
  \caption{Mögliche Beschleunigung der Pioneer-Sonden durch die Gravitation des Kuipergürtels \cite{Anderson2002}}
\label{fig:kuiper}
\end{figure} 


\bigskip

\subsection{Sondeninterne Fehlerquellen}\label{intern}

\subsubsection{Radiowellenr\"ucksto{\ss}}

Die Pioneer-Sonden haben eine Sendeleistung von 8 W, die wie folgt
abgestrahlt wird:
\begin{equation}
P_{\mathit{SL}}=\int _{0}^{\theta _{\mathit{max}}}\sin \theta \rho
(\theta )d\theta
\end{equation}
Dabei ist $\rho (\theta )$ die Leistungsverteilung.

Die Beschleunigung der Sonde durch die Radiowellen l\"asst sich
berechnen mit:
\begin{equation}
b_{\mathit{SL}}=\frac{\beta P_{\mathit{SL}}}{\mathit{Mc}}
\end{equation}
wobei $b_{\mathit{SL}}$ von der Erde weg zeigt. Dabei ist $\beta $ die
partielle Komponente des Strahlungsmoments, welche in entgegengesetzter
Richtung zu $a_{p}$ zeigt\cite{Anderson2002}:
\begin{equation}
\beta =\frac{1}{P_{\mathit{SL}}}\int _{0}^{\theta _{\mathit{max}}}\sin
\theta \cos \theta \rho (\theta )d\theta
\end{equation}
Messungen  zeigten\cite{Anderson2002}, dass man den Strahl als konisch annehmen kann.
Mit einem Gesamt\"offnungswinkel von $\theta =3,75{}^{\circ}$ ist
$\beta =0,99\pm 0,01$ und damit $w_{\mathit{SL}}=1,10\cdot
10^{-8}\mathit{cm}/s^{2}$.

Mit einem Fehler der Sendeleistung und der Ungenauigkeit der Masse
erh\"alt man als Ergebnis f\"ur die Beschleunigung der Sonden durch die
Sendeleistung
\begin{equation}
a_{\mathit{SL}}=-1,10\pm 0,11\cdot 10^{-8}\mathit{cm}/s^{2}
\end{equation}


\bigskip

\subsubsection{Ungleichm\"a{\ss}ige Abstrahlung der RTGs}

W\"ahrend dem Flug zum Jupiter war die Sonde relativ nahe an der Sonne.
Eine Erkl\"arung f\"ur die Beschleunigung k\"onnte nun folgende sein:
Die Fl\"achen der RTGs\footnote{RTG steht für radioisotope thermoelectric generator, zu Deutsch Radionuklidbatterie und ist im äußeren Planetensystem die einzig sinnvolle Energiequelle. Die Energieversorgung einer Raumsonde durch Solarzellen ist nicht praktikabel, da die Solarzellen enorme Flächen bräuchten um genug Leistung bereit zu stellen. In einem RTG zerfällt auf natürliche Weise ein $\alpha$-Strahler und erzeugt durch den Stoss der ausgesendeten Heliumkerne an anderen Atomen Wärme. Diese W\"arme nehmen thermoeletrische Bauelemente auf und wandeln sie direkt in elektrische Energie um.}, die der Sonne zugewandt waren, haben eine
h\"ohere Strahlendosis durch den Sonnenwind erfahren als die der Sonne
abgewandten. Dies geschah gleichzeitig mit dem Auftreffen von interplanetarem Staub auf die sonnenabgewandten Seiten. Dadurch entstand ein
Strahlungsgradient und somit eine ungleichm\"a{\ss}ige Abstrahlung der
RTGs. Durch die spezielle Bauweise der K\"uhlrippen aus einer
Magnesiumlegierung, die mit einem Zirkonium-Natrium Silicat beschichtet
war, besitzen die K\"uhlrippen einen hohen Emissionskoeffizienten
von ca. 0,9 und einen geringen Absorptionskoeffizienten von ca. 0,2. Um
nun $a_{p}$ hervorzurufen m\"usste es eine unterschiedliche
Abstrahlung in Front/Heck- Ausrichtung von 10\% gegeben haben. Nach
unserem Kenntnisstand des Sonnenwindes und des interplanetaren Staubs
ist es nicht m\"oglich, dass diese zwei Ph\"anomene einen derart
gro{\ss}en Gradienten mit dem richtigen Vorzeichen erzeugen k\"onnen.
Dies wurde durch visuelle Beweise der Voyager Sonden gest\"utzt\cite{Anderson2002}.

W\"ahrend den Flyby Man\"overn wurden die Sonden jedoch sehr hoher
Strahlung ausgesetzt, die eine Besch\"adigung der RTGs zur Folge haben
k\"onnte. Man h\"atte also w\"ahrend eines Flyby Man\"overs einen
Anstieg in der thermischen Emission beobachten m\"ussen. Da f\"ur die
Oberfl\"achenenergiebelastung $F\alpha T^{4}$ gilt, h\"atte man eine
Temperaturdifferenz an den K\"uhlrippen beobachten m\"ussen. Die
Durchschnittstemperatur der K\"uhlrippen lag bei ca. 440 K \cite{Anderson2002}. Um
$a_{p}$ zu erkl\"aren m\"usste ein Unterschied von 10\% bzw. $\approx$12,2 K
herrschen. Dieser Unterschied wurde jedoch nicht beobachtet und somit
kann auch dies als Ursache f\"ur die Anomalie ausgeschlossen werden. Um
diesem Effekt dennoch gerecht zu werden, wird f\"ur die Fehlerrechnung
eine unterschiedliche Abstrahlung von 1\% angenommen. Bei einer
thermischen Leistung von 2000 W liegt eine unterschiedliche Abstrahlung
von 10 W in Front/Heck-Ausrichtung vor. Da die K\"uhlrippen in einem
Abstand von 30{\textdegree} angebracht sind, ergibt sich aus
$[10W]\cdot \int _{0}^{\pi }[\sin \Phi ]d\Phi /\pi \approx
6,12W$\cite{Anderson2002} (da 4 der 12 Rippen senkrecht und parallel zur Flugrichtung
stehen) eine Ungenauigkeit von

\begin{equation}
\sigma _{\mathit{UA}}=0,85\cdot 10^{-8}\mathit{cm}/s^{2}
\end{equation}


\bigskip

\subsubsection{ Aussto{\ss} von Helium aus den RTGs}

Eine weitere Erkl\"arung der anomalen Beschleunigung ist, dass durch den
$\alpha ${}-Zerfall von Pu-238 Helium aus den RTGs austritt. Die RTGs
der Pioneer Sonden wurden so konstruiert, dass das Helium aus der
Hitzequelle in den thermoelektrischen Konverter diffundieren kann. Der
Konverter ist mit einem Dichtungsring(O-Ring) abgeschlossen, der es dem Helium
allerdings erm\"oglicht in den Weltraum zu entweichen. Im gesamten Brennstoff \footnote{Klumpen radioaktiver Müll} der RTGs sind 5,8 kg Pu-238 enthalten\cite{Anderson2002}. Mit einer
Halbwertszeit von 87,74 Jahren werden pro Jahr ca. 0,77 g Helium mit einer
Temperatur von 433 K ausgesto{\ss}en. Dies entspricht nach $E_kin=3/2 kT$ einer
Geschwindigkeit von 1,22 km/s. Mit der Raketengleichung
\begin{equation}
a(t)=-v(t)\frac{d}{\mathit{dt}}[\ln M(t)]
\end{equation},
unser nominellen Pioneer 10 Masse von 241 kg und der Annahme, dass das
Helium die RTGs ungerichtet verl\"asst, erh\"alt man eine
Beschleunigung von $1,16\cdot 10^{-8}\mathit{cm}/s^{2}$.

Der Gasaustoss ist jedoch nicht ungerichtet, sondern, da der O-Ring an
der Seite der RTGs angebracht ist, in Richtung Sonne. Durch die
K\"uhlrippen der RTGs wird au{\ss}erdem noch Helium elastisch
reflektiert, sodass man auf eine Beschleunigung aufgrund des
Heliumaussto{\ss}es von
\begin{equation}
a_{\mathit{He}}=-(0,15\pm 0,16)\cdot 10^{-8}\mathit{cm}/s^{2}
\end{equation}
kommt\cite{Anderson2002}.


\bigskip

\subsubsection{Gasleck im Antriebssystem}

Da es keine perfekten Ventile gibt, muss man immer mit einem leichten
Gasaustritt im Antriebssystem rechnen. Einige Sonden\cite{Anderson2002} haben
deswegen Beschleunigungen von bis zu $10^{-7}\mathit{cm}/s^{2}$
erfahren. Das Antriebssystem der Pioneers besteht aus drei Paar
Korrekturd\"usen, die im 120{\textdegree} Abstand um die Parabolantenne
angebracht sind. Von diesen drei Paaren sind zwei parallel zur
Sonnenl\"angsachse ausgerichtet um die Pr\"azession der Parabolantenne
zu steuern. Ein Paar ist tangentiell zur Antenne positioniert um die
Rotation zu steuern. Aus den Ver\"anderungen der Rotationsrate in den
Intervallen i=I,II,III ist die Kraft eines Gaslecks mit einem Hebelarm
von R=1,37 m und einem Tr\"agheitsmoment von
$I_{z}=588,3\mathit{kg}\cdot\ m^{2}$
\begin{equation}
F_{\theta }=\frac{I_{z}\theta }{R}=(2,57;12,24;1,03)\cdot
10^{2}N
\end{equation}

Um nun die Kraft der zwei anderen Korrekturd\"usen abzusch\"atzen, nimmt
man an, dass diese das gleiche Gasleck haben, wie die
Rotationskorrekturd\"usen. So kann man nun annehmen\cite{Anderson2002}, dass
\begin{equation}
F_{\mathit{GL}}\simeq \pm \sqrt{(2)}F_{\theta }=(\pm 3,64;\pm
17,31;\pm 1,46)\cdot 10^{2}N
\end{equation}
ist. Unter der weiteren Annahme dass die Fehler normalverteilt sind,
liegt der Fehler f\"ur Pioneer 10 bei
\begin{equation}
\sigma _{\mathit{GL}}=\pm 0,56\cdot 10^{-8}\mathit{cm}/s^{2}
\end{equation}
Dies ist die gr\"o{\ss}te Ungenauigkeit, aber jedoch nicht gro{\ss}
genug um die Pioneer-Anomalie zu erkl\"aren.


\bigskip

\subsubsection{R\"ucksto{\ss} durch thermische Abstrahlung}\label{Hitze}

In der urspr\"unglichen Arbeit von 2002 \cite{Anderson2002} standen den Autoren nur
begrenzte Telemetriedaten zur Verf\"ugung. So haben sie einen Wert
f\"ur den R\"ucksto{\ss} der thermischen Abstrahlung und der
ungleichm\"a{\ss}igen Abk\"uhlung der Sonden von
$a_{\mathit{tA}/\mathit{Ak}}=0,55\pm 0,73\cdot
10^{-8}\mathit{cm}/s^{2}$ errechnet \cite{Anderson2002}[Tabelle X]. 2003 wurde jedoch
eine Arbeit ver\"offentlicht, die eine gerichtete thermische
Abstrahlungsleistung von 52 W angab. Dieser hohe Wert wurde durch
andere Berechnungen, wie die Annahme eines Lambertstrahlers(ca. 45 W)
oder eine Finite Elemente Methode(ca. 48 W), gest\"utzt\cite{Turyshev2010}. Diese Berechnungen zeigen, dass die Kraft der
thermischen Strahlung vollkommen untersch\"atzt wurde. Leider waren
auch die neuen Werte nur auf groben Absch\"atzungen der thermischen und
elektrischen Leistung an Board der Sonden gest\"utzt. Au{\ss}erdem
machen sie keine Aussagen \"uber den zeitlichen Verlauf der
Abstrahlung. Da f\"ur beide Pioneer Missionen mittlerweile alle
Telemetriedaten vorliegen wurden neue Untersuchungen der thermischen
Abstrahlung der Sonden durchgef\"uhrt.

Hier wurden zwei Hauptthermalquellen ausgemacht: Zum einen die RTGs und
zum anderen die elektrischen Ger\"ate an Board.


\bigskip
Da sich alle 4 RTGs einer jeden Pioneer Sonde zeitlich gleich verhalten,
kann man sie als eine Hitzequelle beschreiben. Die Leistung der RTGs
verh\"alt sich nach dem Zerfallsgesetz mit
\begin{equation}
P_{\mathit{rtg}}(t)=2^{\frac{-(t-t_{0})}{T}}P_{\mathit{rtg}}(t_{0})
\end{equation}
mit dem Zeitpunkt $t_0$ an dem die
Leistung $P_{\mathit{rtg}}(t_{0})=(650\pm 1)W/\mathit{RTG}$ gemessen
wurde und der Halbwertszeit von Pu-238 von T=87,74 a. Da man f\"ur die
abgenommene elektrische Leistung die genauen Daten kennt (Abb.\ref{fig:hitze}),
l\"asst sich die W\"armeleistung schreiben als
\begin{equation}
B_{\mathit{rtg}}(t)=P_{\mathit{rtg}}(t)-P_{\mathit{el}}(t)
\end{equation}


\bigskip

\begin{figure}[htbn]
\begin{center}
\noindent    
\psfig{figure=images/hitze,width=0.8\linewidth,height=\textheight,keepaspectratio}
\end{center}
\vskip -10pt
  \caption{Hitzeentwicklung der RTGs (rote Datenpunkte, Skala auf der linken Seite) und der elektronischen Geräte (grüne Datenpunkte, Skala auf der rechten Seite) über die Funktionsdauer von Pioneer 10 \cite{Turyshev2010}}
\label{fig:hitze}
\end{figure} 


\bigskip

In Abb. \ref{fig:hitze} ist au{\ss}erdem noch die elektrische Leistung aufgetragen,
die von ca. 160 W am Start der Mission langsam auf ca. 60 W abfiel, als
Pioneer 10 das letzte Signal sendete. Teilweise wurde die entnommene
elektrische Leistung aus den RTGs in den Ger\"aten unterschiedlich stark in Hitze umgewandelt
und abgestrahlt. Obwohl die Verteilung der Hitzequellen innerhalb der
Sonde nicht gleichm\"a{\ss}ig war, blieb die Temperaturverteilung in
der Sonde als Funktion der Zeit linear\cite{Turyshev2010}. Somit kann
man die Hitzeabstrahlung aufgrund von elektrischen Ger\"aten als eine
Hitzequelle behandeln. Die Daten hierf\"ur liefert die Telemetrie(Abb.\ref{fig:hitze}).

Die Berechnung der Kraft auf die Sonden gestaltet sich etwas einfacher,
da durch die Spinstabilisierung die Kraft \textit{F }nur entlang des
Einheitsvektors der Rotationsachse \textit{\bf s} berechnet werden muss:
\begin{equation}
F=\frac{1}{c}(\xi _{\mathit{rtg}}B_{\mathit{rtg}}+\xi
_{\mathit{el}}B_{\mathit{el}}){\bf s}
\end{equation}

Die Faktoren $\xi _{\mathit{rtg}}$ und $\xi _{\mathit{el}}$ lassen
sich aus der Geometrie und den thermischen Eigenschaften der Sonden
berechnen\cite{Turyshev2010}.

Die Berechnung dieser Kraft ist Bestandteil aktueller Studien zur
Pioneer-Anomalie. Auf jeden Fall belegen sie den vollkommen
untersch\"atzten Einfluss der thermischen Abstrahlung auf die
Beschleunigung. Mithilfe der neuen, kompletten Telemetriedaten ist es
nun m\"oglich sehr gute Finite Elemente Modelle der Pioneer Sonden im
Bezug auf die Hitzeverteilung zu erstellen. Sollte die Anomalie, wenn
auch nur teilweise, von dieser thermischen Kraft erkl\"art werden, muss
man sich erneut Gedanken \"uber die Konstanz von $a_{p}$ machen.


\bigskip

\subsection{ Fehlertabelle und endg\"ultiges Ergebnis}

In Kapitel \ref{extern} und \ref{intern} haben wir gesehen, dass die meisten Effekte, die 
die Pioneer Sonden erfahren, nur zu einem sehr kleinen Teil Einfluss auf die
Beschleunigung haben. Wir haben in dieser Arbeit nur die wichtigsten
Fehlerquellen zusammengefasst. Diese und andere Fehlerquellen, ihre
Werte und Ungenauigkeiten sind in nachfolgender Tabelle aufgelistet.


\bigskip
\begin{table}[htbn]\label{fehler}
\newcommand{\mc}[3]{\multicolumn{#1}{#2}{#3}}
\centering
\begin{tabular}{|c|c|c|}
\hline
Beschreibung &
Wert $\cdot (10^{-8}) cm/s^2$ &
$\sigma\ in\ (10^{-8}) cm/s^2$\\ \hline
\mc{3}{|c|}{\bf 1. Externe Fehlerquellen} \\ \hline
Strahlungsdruck der Sonne &
\raggedleft -0,03 &
 0,01\\ \hline
Sonnenwind &
~
 &
 {\textless}10\^{}-3\\ \hline
Sonnencorona &
~
 &
 0,02\\ \hline
Elektromagnetische Lorentzkr\"afte &
~
 &
 {\textless}10\^{}-4\\ \hline
Einfluss der Gravitation des Kuiperg\"urtel &
~
 &
 0,03\\ \hline
Einfluss der Erdorientierung &
~
 &
 0,001\\ \hline
Mechanische und Phasenstabilit\"at der DSN Antennen &
~
 &
 {\textless}0,001\\ \hline
Phasenstabilit\"at und Uhren &
~
 &
 {\textless}0,001\\ \hline
DSN Antennen Orte &
~
 &
 {\textless}10\^{}-5\\ \hline
Troposph\"are und Ionosph\"are &
~
 &
 {\textless}0,001\\ \hline
~
 &
~
 &
~
\\ \hline
\mc{3}{|c|}{\bf 2. Sonden interne Fehlerquellen}\\ \hline
R\"ucksto{\ss} der Radiowellen &
\raggedleft -1,1 &
 0,11\\ \hline
Hitze, reflektiert von der Sonde &
\raggedleft {}0,55 &
 0,55\\ \hline
Ungleichm\"a{\ss}ige Hitzeabstrahlung der RTGs &
~
 &
 0,85\\ \hline
Ungleichm\"a{\ss}ige Abk\"uhlung der Sonden &
~
 &
 0,48\\ \hline
Heliumaussto{\ss} aus den RTGs &
\raggedleft -0,15 &
 0,16\\ \hline
Gaslecks &
~
 &
 0,56\\ \hline
Unterschiede zwischen den Sonden &
\raggedleft -0,17 &
 0,17\\ \hline
~
 &
~
 &
~
\\ \hline
\mc{3}{|c|}{\bf 3. Rechnerische Fehlerquellen}\\ \hline
Numerische Stabilit\"at der Finite Elemente Methode &
~
 &
 0,02\\ \hline
Genauigkeit der Fehlerabsch\"atzung und Modelle &
~
 &
 0,13\\ \hline
Mismodellierung von Man\"overn &
~
 &
 0,01\\ \hline
Mismodellierung der Sonnencorona &
~
 &
 0,02\\ \hline
Jahres- und Tagesschwankungen &
~
 &
 0,32\\ \hline
\end{tabular}
\caption{Fehlertabelle}
\label{tab:fehler}
\end{table}

\noindent Das Ergebnis der anormalen Beschleunigung von Pioneer 10 und 11 liegt
bei\cite{Anderson2002}
\begin{equation}
a_{p(\mathit{gemessen})}=(7,84\pm 0,01)\cdot
10^{-8}\mathit{cm}/s^{2}\ .
\end{equation}
Bezieht man nun die Werte der Fehler und deren Ungenauigkeit von Kapitel
\ref{extern} und \ref{intern} mit ein, so erh\"alt man mit den Werten der Fehlertabelle
und mit
\begin{equation}
a_{p}=a_{p(\mathit{gemessen})}-(w_{p}\pm \sigma _{p}) \ ,
\end{equation}
wobei $w_{p}$ der Wert und $\sigma _{p}$ die jeweilige Ungenauigkeit
ist, den bekannten Wert f\"ur $a_{p}=(8,74\pm 1,33)\cdot
10^{-8}m/s^{2}$.

\FloatBarrier