
\section{Geschichte}

Im Februar 1969 genehmigte die NASA ( National Aeronautics and Space
Administration) ein Programm, um den Asteroideng\"urtel, das
interplanetare Medium zwischen Mars und Jupiter, die \"au{\ss}eren
Planeten und Fly-By Man\"over zu erforschen. Hierzu wurden zwei
baugleiche Sonden Pioneer F (Pioneer 10 Mission) und Pioneer G (Pioneer
11 Mission) zum Jupiter gebracht. Die Pioneer 10 Mission startete am 2.
M\"arz 1972 und wurde dann auf ca. 14,4 km/s beschleunigt. Die Sonde
durchflog im Juli 1972 unbeschadet den Asteroideng\"urtel und erreichte
am 4. Dezember 1973 den Jupiter. Hier nutzte man ein Fly-By Man\"over
um die Sonde auf eine heliozentrische Fluchtgeschwindigkeit von 11,322
km/s (Gesamtgeschwindigkeit 36,7 km/s)zu beschleunigen um das
Sonnensystem in Richtung des Sterns Aldebaran (Laut Zeitplan sollte die
Sonde den Stern in ungef\"ahr 2 Millionen Jahren erreichen\cite{Nieto2007}) zu verlassen. Pioneer
11 startete 13 Monate sp\"ater, am 6. April 1973, da die NASA mit
Pioneer 10 erst herausfinden wollte, ob eine Durchquerung des
Asteroideng\"urtels \"uberhaupt m\"oglich ist. Ihre Bahn f\"uhrte
Pioneer 11 ebenfals Richtung Jupiter, den sie am 2. Dezember 1974
erreichte. Das dort durchgef\"uhrte Fly-By Man\"over brachte sie auf
eine Bahn, die Pioneer 11 zun\"achst wieder innerhalb der Jupiter-Bahn
f\"uhrte, um dann aber am 1. September 1979 den Saturn zu erreichen
(Abb. \$1) In einem weiteren Fly-By Man\"over, bei dem die Sonde die
Ringe des Saturns unbeschadet durchquert hat, wurde sie auf eine
asymptotische Fluchtgeschwindigkeit von 10,450 km/s gebracht. Pioneer
11 steuert auf die Konstellation Aquila zu, wo sie in ungef\"ahr 4
Millionen Jahren eintreffen wird. Die Relationen der Flugbahnen der
Sonden Pioneer 10 und 11, sowie Voyager 1 und 2 sind in Abb. \$2 zu
erkennen.


\bigskip

Abb. \$1 und \$2


\bigskip

Obwohl Pioneer 10 und 11 nur auf eine Betriebszeit von 21 Monate
ausgelegt waren, sendete Pioneer 10 Messdaten bis zum 27. April 2002.
Das letzte Signal von Pioneer 10 erreichte die Erde am 23. Januar 2003.
Das letzte Signal von Pioneer 11 wurde jedoch deutlich fr\"uher, am 24.
November 1995 empfangen, da durch das zweite Fly-By Man\"over am Saturn
sehr viel mehr Leistung ben\"otigt wurde.

Zu den o.g. Missionszielen geh\"orte vor allem unter dem Ziel der
Erforschung der \"au{\ss}eren Planeten die Suche nach dem
{\quotedblbase}Planeten X``, der damals jenseits von Neptun vermutet
wurde. Um das schwache Gravitationsfeld dieses omin\"osen Planeten
nachzuweisen und um m\"oglichst nahe an Jupiter und Saturn vorbei zu
fliegen, ben\"otigten die Pioneer-Sonden eine sehr genaue Navigation.
Dabei wurden von einer Bodenstation des Deep Space Network DSN (in
Goldstone/USA, Madrid/Spanien, Canberra/Australien) Radiowellen mit
einer wohldefinierten Frequenz zur Sonde geschickt. Die Pioneers
sendete dieses Signal mit einer um den Faktor 240/221 konvertierten
Frequenz wieder zur Erde zur\"uck\cite{Dittus2006}.
Diese genaue Navigation erlaubte schlie{\ss}lich die Entdeckung der
Pioneer-Anomalie. 

Damit die Parabolantenne immer auf die Erde gerichtet blieb, musste
die Sonde vor allem nach Vorbeifl\"ugen an gro{\ss}en Planeten neu
ausgerichtet werden. Hierzu wurden kleine Triebwerke f\"ur eine kurze
Zeit gez\"undet. Alle weiteren St\"orfaktoren auf die Flugbahn von
Pioneer 10 und 11 wurden mit einer Eigenrotation der Sonden um die
Symmetrieachse der Parabolantenne von 4 bis 7 U/min
ausgeglichen\cite{Dittus2006} \cite{Nieto2007}.

Durch die genaue Navigation und die Verminderung von Fehlern,
bemerkte man Anfang der 80-er eine unvorhergesehene Beschleunigung von 
$(8,74\pm 1,33)\cdot 10^{-8}\mathit{cm}/s^{2}$ \cite{Anderson2002} in Richtung der Sonne. 

Diese Beschleunigung wurde schlie{\ss}lich zur Pioneer-Anomalie, deren
Ursache bis heute nicht bekannt ist. 