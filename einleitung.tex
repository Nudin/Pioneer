\section{Geschichte}

\begin{figure}[htbn]
\begin{center}
\noindent    
\psfig{figure=images/pio-craft,width=\linewidth,height=\textheight,keepaspectratio}
\end{center}
\vskip -10pt
  \caption{Schemazeichnung der baugleichen Pioneer-Sonden\cite{Turyshev2010}}\label{fig:pioneer}
\end{figure} 

\begin{figure}[htbn]
\begin{center}
\noindent    
\psfig{figure=images/pioneer_path,width=0.8\linewidth,height=\textheight,keepaspectratio}
\end{center}
\vskip -10pt
  \caption{Eine Aufsicht der Flugbahnen von Pioneer 10 und 11, sowie von den Voyager-Sonden 1 und 2; betrachtet vom nördlichen Pol der Ekliptik\cite{Anderson2002}}
\label{fig:flugbahn}
\end{figure} 


Im Februar 1969 genehmigte die NASA (National Aeronautics and Space
Administration) ein Programm um den Asteroideng\"urtel, das
interplanetare Medium zwischen Mars und Jupiter, die \"au{\ss}eren
Planeten und Flyby Man\"over zu erforschen. Hierzu wurden zwei
baugleiche Sonden der Pioneer Reihe zum Jupiter gebracht. Die Pioneer 10 Mission startete am 2.
M\"arz 1972 und wurde dabei auf ca. 14,4 km/s beschleunigt. Die Sonde
durchflog im Juli 1972 unbeschadet den Asteroideng\"urtel und erreichte
am 4. Dezember 1973 den Jupiter. Hier nutzte man ein Flyby Man\"over
um die Sonde auf eine heliozentrische Fluchtgeschwindigkeit von 11,322
km/s zu beschleunigen um das Sonnensystem in Richtung des Sterns Aldebaran zu verlassen. Laut Zeitplan sollte die
Raumsonde den Stern in ungef\"ahr 2 Millionen Jahren erreichen\cite{Nieto2007}. Pioneer
11 startete 13 Monate sp\"ater (am 6. April 1973), da die NASA mit
Pioneer 10 erst herausfinden wollte, ob eine Durchquerung des
Asteroideng\"urtels \"uberhaupt m\"oglich ist. Ihre Bahn f\"uhrte
Pioneer 11 ebenfals Richtung Jupiter, den sie am 2. Dezember 1974
erreichte. Das dort durchgef\"uhrte Flyby Man\"over brachte sie auf
eine Flugbahn, die zun\"achst wieder innerhalb der Jupiter-Bahn
verlief, um dann aber am 1. September 1979 den Saturn zu erreichen
(Abb. \ref{fig:flugbahn}). In einem weiteren Flyby Man\"over, bei welchem die Sonde die
Ringe des Saturns unbeschadet durchquerte, wurde sie auf eine
asymptotische Fluchtgeschwindigkeit von 10,450 km/s beschleunigt. Pioneer
11 steuert seitdem auf die Konstellation Aquila zu, wo sie in ungef\"ahr 4
Millionen Jahren eintreffen wird.

\bigskip

Obwohl Pioneer 10 und 11 nur auf eine Betriebszeit von 21 Monate
ausgelegt waren, sendete Pioneer 10 Messdaten bis zum 27. April 2002.
Das letzte, schwache Signal von Pioneer 10 erreichte die Erde am 23. Januar 2003.
Das letzte Signal von Pioneer 11 wurde jedoch deutlich fr\"uher, am 24.
November 1995 empfangen, da durch das zweite Flyby Man\"over am Saturn
sehr viel mehr Leistung an Board ben\"otigt wurde.

Neben den o.g. Missionszielen geh\"orte vor allem unter dem Ziel der
Erforschung der \"au{\ss}eren Planeten die Suche nach dem
{\quotedblbase}Planeten X``, der damals jenseits von Neptun vermutet
wurde. Um das schwache Gravitationsfeld dieses omin\"osen Planeten
nachzuweisen und um m\"oglichst nahe an Jupiter und Saturn vorbei zu
fliegen, ben\"otigten die Pioneer-Sonden eine sehr genaue Navigation.
Dabei wurden von einer Bodenstation des Deep Space Network DSN (s. Kap. \ref{messung}) Radiowellen mit
einer wohldefinierten Frequenz zur Sonde geschickt. Die Pioneers
sendeten dieses Signal mit einer um einen konstanten Faktor konvertierten
Frequenz wieder zur Erde zur\"uck\cite{Dittus2006}. Die aus der Bewegung der Sonde resultierende doppelte Dopplerverschiebung ermöglicht eine sehr genaue Bestimmung der Geschwindigkeit der Sonde. Diese genaue Navigation erlaubte schlie{\ss}lich die Entdeckung der
Pioneer-Anomalie.

Damit die Parabolantenne immer auf die Erde gerichtet blieb, musste
die Sonde vor allem nach Vorbeifl\"ugen an gro{\ss}en Planeten neu
ausgerichtet werden. Hierzu wurden kleine Triebwerke f\"ur eine kurze
Zeit gez\"undet. Weiteren St\"orfaktoren auf die Flugbahn von
Pioneer 10 und 11 wurden mit einer Eigenrotation der Sonden um die
Symmetrieachse der Parabolantenne von 4 bis 7 U/min
ausgeglichen\cite{Nieto2007}\cite{Dittus2006} .

Durch die genaue Navigation und die Verminderung von Fehlern,
bemerkte man Anfang der 80-er eine unvorhergesehene Beschleunigung von
$(8,74\pm 1,33)\cdot10^{-8}\mathit{cm}/s^{2}$ \cite{Anderson2002} in Richtung Sonne.

Diese Beschleunigung wurde schlie{\ss}lich Pioneer-Anomalie genannt, deren
Ursache bis heute nicht bekannt ist. 