\section{Dunkle Energie}\label{De}



Schon fr\"uh fiel auf, dass f\"ur den Wert der Pioneer-Anomalie
$a_{p}\simeq \mathit{cH}_{0}$ gilt, wenn c die
Vakuumlichtgeschwindigkeit und $H_0$ die
derzeitige Hubble-Konstante ist. Deshalb wurden Vermutungen
laut, dass die Anomalie mit der Ausdehnung des Universums und somit mit
der Dunklen Energie in Zusammenhang steht\cite{Turyshev2010}. Die Idee hierzu ist,
dass die Anomalie gar keine echte Beschleunigung ist, sondern nur das
Doppler-Signal durch die Ausdehnung des Universums beeinflusst wird.
Die Grundfrage hier war also ob die Dunkle Energie einen messbaren
Einfluss auf elektromagnetische Wellen hat.



Die Sonde bewegt sich mit der Geschwindigkeit $v$. Demnach w\"are
die Beschleunigung durch das sich ausdehnende Universum $a_H$:

\begin{equation*}
a_{H}=v\cdot H_{0}=\frac{v}{c}\cdot c\cdot H_{0}
\end{equation*}

Man sieht deutlich, dass die Beschleunigung
$a_H$ um den Faktor v/c kleiner w\"are,
als die beobachtete $a_p$
Au{\ss}erdem ist $c\cdot H_{0}$ f\"ur
$H_{0}=73,2\frac{\mathit{km}}{s\cdot \mathit{Mpc}}$ nur eine
Ann\"aherung an den tats\"achlichen Wert der Anomalie. Um den exakten
Wert f\"ur $a_p$ zu erhalten br\"auchte
man eine Hubble-Konstante mit einem Wert von $H_{0}=95\pm
14\frac{\mathit{km}}{s\cdot \mathit{Mpc}}$, was weit außerhalb des Fehlerbereichs von $H_0$ liegt. Ein weiteres
Argument gegen den Einfluss der Dunklen Energie ist, dass die
Beschleunigung in Richtung Sonne zeigt. W\"are die Dunkle Energie
tats\"achlich die Ursache, w\"urden die Pioneer-Sonden von der Sonne
bzw. der Erde weg beschleunigt. Im Doppler-Signal w\"urde sich das in
einer Rotverschiebung, anstatt in der beobachteten
Blauverschiebung \"au{\ss}ern.


