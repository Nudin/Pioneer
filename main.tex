\documentclass[a4paper,10pt]{article}
\usepackage[utf8x]{inputenc}

\newcommand{\rem}[1]{}


%opening
\title{Die Pioneer-Anormalie}
\author{Judith, Michi, Flo}

\begin{document}

\maketitle

% \begin{abstract}
% 
% \end{abstract}

\section{Geschichte/Einführung/Die Sonden}
Im Februar 1969 genehmigte die NASA ( National Aeronautics and Space Administration) ein Programm, um den Asteoriedengürtel,
das interplanetare Medium zwischen Mars und Jupiter und die äußeren Planeten zu erforschen.
Hierzu wurden zwei baugleiche Sonden Pioneer F (Pioneer 10 Mission) und Pioneer G (Pioneer 11 Mission) zum Jupiter gebracht.
Die Pioneer 10 Mission startete am 2. März 1972

Die Sonden Pioneer 10 und Pioneer 11 waren die ersten Missionen, die die äußeren Planeten unseres Sonnensystems erforschen sollten. 


\section{Die Anormalie}
\subsection{Frequenzmessung}

Die Geschwindigkeitsmessung der Pioneersonden erfolgte über die Zwei-Wege-Dopplerverschiebung von Radiowellen.
Von den Bodenstationen des Deep Space Networks wurden Radiowellen mit bekannter Frequenz zum Satelliten gesendet (uplink).
Der Satelit empfängt das Signal dopplerverschoben:
\begin{equation}
 \nu_R = \frac{1}{\sqrt{1-\frac{v^2}{c^2}}}(1-\frac{v}{c})\nu_E
\end{equation}
und antwortet mittels einer 8-Watt Sendeanlage und eines Transponders
mit einer um den festen Faktor $ \frac{240}{221} $ verschobenen Frequenz:
\begin{equation}
\nu'_R = \nu_R\frac{240}{211}
\end{equation}
Beim Rückweg wird das Signal ein zweites mal identisch dopplerverschoben.
Das Empfangene Signal ist also zweifach doppleverschoben und um den Faktor $\frac{240}{221}$ verschoben.
\begin{equation}
 \nu'_E = \frac{1}{\sqrt{1-\frac{v^2}{c^2}}}(1-\frac{v}{c}) \cdot \frac{240}{211}\nu_R \, = \, \frac{1}{1-\frac{v^2}{c^2}}(1-\frac{v}{c})^2 \cdot \frac{240}{211} \nu_E
\end{equation}
Die relative Verschiebung ergibt sich also zu
\begin{equation}
 \frac{\nu'_E-\nu_E}{\nu_E} = \frac{\frac{19}{221}- \frac{461}{221}\frac{v}{c}}{1+\frac{v}{c}}.
\end{equation}
In vielen Quellen wird die konstante Frequenzverschiebung durch die Elektronik der Quelle vernachlässigt, was zur einfacheren Form von
\begin{equation}
 \frac{\nu'_E-\nu_E}{\nu_E} \approx -2\frac{v/c}{1+v/c} \approx -2 \frac{v}{c}
\end{equation}
führt.
Darüber hinaus lässt sich die Entfernung $d$ der Sonde auch durch die Laufzeit $\Delta t$ des Signales bestimmen:
\begin{equation}
 2d = c \Delta t
\end{equation}
Somit hat man zwei voneinander unabhängige Messmethoden, was Konsitenzchecks,
Fehlerminimierung und Ausschluss einiger Phänomenologischer Fehler ermöglicht.

\subsection{Theoretische Berechnungen}
Die ursprüngliche Navigationsberechnung erfolgte durch das Orbital Determination Program (ODP) des JPL.
Hierbei wird die Bahn im Rahmen des relativistischen Einstein-Infeld-Hoffmann-Modells (EIH Modell)
bis einschließlich zur Ordnung $(\frac{v}{c})^4$ berechnet. Berücksichtigt wurden dabei die Sonne,
die 9 Planeten als Punktmassen, der Mond, die größten Asteroiden (ca. 0,2 Erdmassen) und Kometen, Platentektonik der Erde, uvm.
%Zu benutzen: Anderson 2002 + Physikjournal + Vorträge
Eine unabhängige Überprüfung erfolgte durch durch den CHASMP/POEAS-Code des Aerospace Corporation.
Eine dritte bestätigung folge später durch einen von C. Markward (Goddard Space Flight Center, GFSC) geschriebenen Code.
Im Jahr 2008 entwickelte Groupe Anomalie Pioneer (GAP) eine eigene Software namens ODYSSEY,
mit welcher die Anomalie ebenfals bestätigt werden konnte.
Und durch Oslo?


\end{document}
