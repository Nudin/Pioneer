\documentclass[a4paper,10pt]{article}
\usepackage[utf8x]{inputenc}
\usepackage[ngerman]{babel}
\usepackage{cite}

\newcommand{\rem}[1]{}


%opening
\title{Die Pioneer-Anormalie}
\author{Judith Selig, Michael F. Schönitzer, Florian Schlagintweit}

\begin{document}

\maketitle

% \begin{abstract}
% 
% \end{abstract}

\section{Einleitung}
Im Februar 1969 genehmigte die NASA ( National Aeronautics and Space Administration) ein Programm, um den
Asteroidengürtel, das interplanetare Medium zwischen Mars und Jupiter, die äußeren Planeten und Fly-By Manöver zu
erforschen. Hierzu wurden zwei baugleiche Sonden Pioneer F (Pioneer 10 Mission) und Pioneer G (Pioneer 11 Mission) zum
Jupiter gebracht. Die Pioneer 10 Mission startete am 2. März 1972 und wurde dann auf ca. 14,4 km/s beschleunigt. Die
Sonde durchflog im Juli 1972 unbeschadet den Asteroidengürtel und erreichte am 4. Dezember 1973 den Jupiter. Hier nutzte
man ein Fly-By Manöver um die Sonde auf eine asymptotische Fluchtgeschwindigkeit von 11,322 km/s (Gesamtgeschwindigkeit
36,7 km/s)zu beschleunigen um das Sonnensystem in Richtung des Sterns Aldebaran (Laut Zeitplan sollte die Sonde diese
Region in ungefähr 2 Millionen Jahren erreichen\cite{Nieto2004}) zu verlassen. Pioneer 11 startete 13 Monate später, am
6. April 1973, da die NASA mit Pioneer 10 erst herausfinden wollte, ob eine Durchquerung des Asteroidengürtels überhaupt
möglich ist. Ihre Bahn führte Pioneer 11 ebenfals Richtung Jupiter, den sie am 2. Dezember 1974 erreichte. Das dort
durchgeführte Fly-By Manöver brachte sie auf eine Bahn, die Pioneer 11 zunächst wieder innerhalb der Jupiter-Bahn
führte, um dann aber am 1. September 1979 den Saturn zu erreichen. In einem weiteren Fly-By Manöver, bei dem die Sonde
die Ringe des Saturns unbeschadet durchquert hat, wurde sie auf eine asymptotische Fluchtgeschwindigkeit von 10,450 km/s
gebracht. Pioneer 11 steuert auf die Konstellation Aquila zu, wo sie in ungefähr 4 Millionen Jahren eintreffen wird. Die
Relationen der Flugbahnen der Sonden Pioneer 10 und 11, sowie Voyager 1 und 2 sind in Abb. 1 zu erkennen.

Abb. 1 (groß)

Obwohl Pioneer 10 und 11 nur auf eine Betriebszeit von 21 Monate ausgelegt waren, sendete Pioneer 10 Messdaten bis zum
27. April 2002. Das letzte Signal von Pioneer 10 erreichte die Erde am 23. Januar 2003. Das letzte Signal von Pioneer 11
wurde jedoch deutlich früher, am 24. November 1995 empfangen, da durch das zweite Fly-By Manöver am Saturn sehr viel
mehr Leistung benötigt wurde.

Zu den o.g. Missionszielen gehörte vor allem unter dem Punkte der Erforschung der äußeren Planeten die Suche nach dem
„Planeten X“, der damals jenseits von Neptun vermutet wurde. Um das schwache Gravitationsfeld dieses ominösen Planeten
nachzuweisen und um möglichst nahe an Jupiter und Saturn vorbei zu fliegen, benötigten die Pioneer-Sonden eine sehr
genaue Navigation. Dabei wurden von einer Bodenstation des Deep Space Network DSN (in Goldstone/USA, Madrid/Spanien,
Canberra/Australien) Radiowellen mit einer wohldefinierten Frequenz zur Sonde geschickt. Die Pioneers sendete dieses
Signal mit einer um den Faktor 240/221 konvertierten Frequenz wieder zur Erde zurück\cite{Dittus2006} Diese genaue
Navigation erlaubte schließlich die Entdeckung der Pioneer-Anomalie. 

Damit die Parabolantenne immer auf die Erde gerichtet blieb, musste die Sonde vor allem nach Vorbeiflügen an großen
Planeten neu ausgerichtet werden. Hierzu wurden kleine Triebwerke für eine kurze Zeit gezündet. Alle weiteren
Störfaktoren auf die Flugbahn von Pioneer 10 und 11 wurden mit einer Eigenrotation der Sonden um die Symmetrieachse der
Parabolantenne von 5 bis 8 U/min ausgeglichen.


\section{Die Anomalie}
\subsection{Frequenzmessung}

Die Geschwindigkeitsmessung der Pioneersonden erfolgte über die Zwei-Wege-Dopplerverschiebung von Radiowellen.
Von den Bodenstationen des Deep Space Networks (in Goldstone/USA, Madrid/Spanien und Canberra/Australien) wurden
Radiowellen mit bekannter Frequenz (S-Band, ~2,11 Ghz) zum Satelliten gesendet (uplink).
Der Satellit empfängt das Signal dopplerverschoben:
\begin{equation}
 \nu_R = \frac{1}{\sqrt{1-\frac{v^2}{c^2}}}(1-\frac{v}{c})\nu_E
\end{equation}
und antwortet mittels einer 8-Watt Sendeanlage und eines Transponders
mit einer um den festen Faktor $ \frac{240}{221} $ verschobenen Frequenz:
\begin{equation}
\nu'_R = \nu_R\frac{240}{211}
\end{equation}
Beim Rückweg wird das Signal ein zweites mal identisch dopplerverschoben.
% ToDo: Verschiebung durch Rotation der Sonde (-> Physik Journal)
Das Empfangene Signal ist also zweifach doppleverschoben und um den Faktor $\frac{240}{221}$ verschoben.
\begin{equation}
 \nu'_E = \frac{1}{\sqrt{1-\frac{v^2}{c^2}}}(1-\frac{v}{c}) \cdot \frac{240}{211}\nu_R \, = \, \frac{1}{1-\frac{v^2}{c^2}}(1-\frac{v}{c})^2 \cdot \frac{240}{211} \nu_E
\end{equation}
Die relative Verschiebung ergibt sich also zu
\begin{equation}
 \frac{\nu'_E-\nu_E}{\nu_E} = \frac{\frac{19}{221}- \frac{461}{221}\frac{v}{c}}{1+\frac{v}{c}}.
\end{equation}
In vielen Quellen wird die konstante Frequenzverschiebung durch die Elektronik der Quelle vernachlässigt, was zur einfacheren Form von
\begin{equation}
 \frac{\nu'_E-\nu_E}{\nu_E} \approx -2\frac{v/c}{1+v/c} \approx -2 \frac{v}{c}
\end{equation}
führt.
Darüber hinaus lässt sich die Entfernung $d$ der Sonde auch durch die Laufzeit $\Delta t$ des Signales bestimmen:
\begin{equation}
 2d = c \Delta t
\end{equation}
Somit hat man zwei voneinander unabhängige Messmethoden, was Konsitenzchecks,
Fehlerminimierung und Ausschluss einiger Phänomenologischer Fehler ermöglicht.

\subsection{Theoretische Berechnungen}
Die ursprüngliche Navigationsberechnung erfolgte durch das Orbital Determination Program (ODP) des JPL.
Hierbei wird die Bahn im Rahmen des relativistischen Einstein-Infeld-Hoffmann-Modells (EIH Modell)
bis einschließlich zur Ordnung $(\frac{v}{c})^4$ berechnet.
Die Gravitation der Sonne, der neun Planeten und des Monds, wurden dabei als isotrope Punktmassen mit dem
PPN-Formalismus (parameterized post-Newtonian formalism) beschrieben\cite{Anderson2002}. Die Gravitation der größten
Asteroiden (ca. 0,2
Erdmassen) und Kometen wurden darüberhinaus gemäß des Newtonschen Gravitationsgesetz mit einberechnet. Auch die
``Terrestrial and lunar figure effects``, die Gezeiten der Erde und die physische Libration wurde mit berücksichtigt.
Die Propagation des Lichtes wurde relativistisch bis zu Ordnung $(\frac{v}{c})^2$ genau berechnet.
Darüber hinaus wurde der Einfluss von Sonnenstrahlung und Sonnenwind berücksichtigt.
% Model of internal non–gravitational forces (->Vortrag) ?
%Zu benutzen: Anderson 2002 + Physikjournal + Vorträge

Eine unabhängige Überprüfung erfolgte durch durch den CHASMP/POEAS-Code des Aerospace Corporation.
Eine dritte Bestätigung folge später durch einen von C. Markward (Goddard Space Flight Center, GFSC) geschriebenen Code.
Im Jahr 2008 entwickelte Groupe Anomalie Pioneer (GAP) eine eigene Software namens ODYSSEY,
mit welcher die Anomalie ebenfalls bestätigt werden konnte.
Eine weitere Bestätigung erfolgte durch den Orbit determination code der Universität Oslo.


\subsection{Die Anomalie}
Ab 1979 beobachtet man zwischen der berechneten und der gemessenen Frequenzverschiebung eine anormale Blauverschiebung.
Die Analyse der Daten von 1987 bis 1998 zeigte eine zeitlich konstante Zunahme dieser Blauverschiebung von
\begin{equation}
  \frac{d\Delta\nu}{dt}=(5,99\pm0,01)\cdot10^{-9}\frac{Hz}{s}
\end{equation}
wobei $\Delta\nu=[\nu_{Messung}-\nu_{Modell}]'_E$. Diese konstante Zunahme der Blauverschiebung wird meist als eine
Beschleunigung in Richtung Sonne interpretiert.
\begin{equation}
  a_{Pioneer}=\frac{dv}{dt}=-\frac{1}{2}\frac{c}{\nu_E}\frac{d\Delta\nu}{dt} = (8,74\pm1,33)\cdot10^{-8}\frac{cm}{s^2}
\end{equation}
\rem{
oder
\begin{equation}
  \Delta\nu=-\nu_E \frac{2a_p t}{c}
\end{equation}
}
Andere Arbeiten mit unterschiedlichen ODPs bestimmten die Beschleunigung zu $(7,70
\pm0,02)\cdot10^{-8}\frac{cm}{s^2}$ (Markwardt,
2002)\footnote{Hier ist jedoch nur der statistische Fehler angegeben}\cite{Markwardt2002} beziehungsweise
$(8,4\pm0,1)\cdot10^{-8}\frac{cm}{s^2}$ (Levy et al., 2008)\cite{Levy2008}.
Wir wollen uns im folgenden jedoch – wie auch praktisch jede Arbeit der Fachliteratur – auf den oben angegebenen von
Anderson et al. berechneten Wert beschränken.

Diese Frequenzverschiebung wurde mit nur maximal 3\% Unterschied bei beiden Pioneer Sonden unabhängig von einander
gefunden. Das anomale Signal variiert über den analysierten Zeitrum um nur maximal 3,5\%. Die Richtung der
Beschleunigung ist mit einer Auflösung von 3° bisher noch recht ungenau bestimmt worden.

%\subsection{Periodisches Verhalten der Anomalie}
%\subsection{Analysen früherer und späterer Daten}


% Noch unterzubringen ist:
%% - S-Band ist für 3-D wenig geeignet. (Route to Understanding)
%% - ''Model of observation stations``?
%% - ...

% ToFix:
%% - Quellen: And -> und
%% - Quellen: Großschreibung nicht ruinieren.

\bibliography{lit}{}
\bibliographystyle{hplain}
\end{document}
