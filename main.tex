\documentclass[a4paper,10pt]{article}
\usepackage[utf8x]{inputenc}

\newcommand{\rem}[1]{}


%opening
\title{Die Pioneer-Anormalie}
\author{Judith, Michi, Flo}

\begin{document}

\maketitle

% \begin{abstract}
% 
% \end{abstract}

\section{Geschichte/Einführung/Die Sonden}
Im Februar 1969 genemigte die NASA ( National Aeronautics and Space Administration) ein Programm, um den Asteoriedengürtel,
das interplanetare Medium zwischen Mars und Jupiter und die äußeren Planeten zu erforschen.
Hierzu wurden zwei baugleiche Sonden Pioneer F (Pioneer 10 Mission) und Pioneer G (Pioneer 11 Mission) zum Jupiter gebracht.
Die Pioneer 10 Mission startete am 2. März 1972

Die Sonden Pioneer 10 und Pioneer 11 waren die ersten Missionen, die die äußeren Planeten unseres Sonnensystems erforschen sollten. 


\section{Die Anormalie}
\subsection{Frequenzmessung}

Die Geschwindigkeitsmessung der Pioneersonden erfolgte über die Zwei-Wege-Dopplerverschiebung von Radiowellen.
Von den Bodenstationen (kontrolliert vom Deep Space Network) wurden Radiowellen mit bekannter Frequenz zum Sateliten gesendet (uplink).
Der Satelit empfänt das Signal dopplerverschoben:
\begin{equation}
 \nu_R = \frac{1}{\sqrt{1-\frac{v^2}{c^2}}}(1-\frac{v}{c})\nu_E
\end{equation}
und antwortet mit einer um den festen Faktor $ \frac{240}{221} $ verschobenen Frequenz:
\begin{equation}
\nu'_R = \nu_R\frac{240}{211}
\end{equation}
Beim Rückweg wird das Signal ein zweites mal identisch dopplerverschoben. Das Empfangene Signal ist also zweifach doppleverschoben und um den Faktor $\frac{240}{221}$ verschoben.
\begin{equation}
 \nu'_E = \frac{1}{\sqrt{1-\frac{v^2}{c^2}}}(1-\frac{v}{c}) \cdot \frac{240}{211}\nu_R \, = \, \frac{1}{1-\frac{v^2}{c^2}}(1-\frac{v}{c})^2 \cdot \frac{240}{211} \nu_E
\end{equation}
Die relative Verschiebung ergibt sich also zu
\begin{equation}
 \frac{\nu'_E-\nu_E}{\nu_E} = \frac{\frac{19}{221}- \frac{461}{221}\frac{v}{c}}{1+\frac{v}{c}}.
\end{equation}
In vielen Quellen wird die konstante Frequenzverschiebung durch die Elektronik der Quelle vereinfacht, was zur einfacheren Form von
\begin{equation}
 \frac{\nu'_E-\nu_E}{\nu_E} \approx -2\frac{v/c}{1+v/c} \approx -2 \frac{v}{c}
\end{equation}
führt.


\rem{

\subsection{Formelpool}
\subsubsection{Ohne Frequenzverschiebung}

\begin{equation}
 \nu' = \frac{1}{\sqrt{1-\frac{v^2}{c^2}}}(1-\frac{v}{c})\nu
\end{equation}

\begin{equation}
 \nu'' = \frac{1}{\sqrt{1-\frac{v^2}{c^2}}}(1-\frac{v}{c})\nu'
\end{equation}

\begin{equation}
 \frac{\nu''-\nu}{\nu} = -2\frac{v/c}{1+v/c} \approx -2 \frac{v}{c}
\end{equation}

\subsubsection{mit Frequenzverschiebung}
\begin{equation}
\nu^{*} = \nu\frac{240}{211}
\end{equation}

\begin{equation}
 \nu'_E = \frac{1}{\sqrt{1-\frac{v^2}{c^2}}}(1-\frac{v}{c}) \cdot \frac{240}{211}\nu_R \, = \, \frac{1}{1-\frac{v^2}{c^2}}(1-\frac{v}{c})^2 \cdot \frac{240}{211} \nu_E
\end{equation}

\begin{equation}
 \frac{\nu'_E-\nu_E}{\nu_E} = \frac{\frac{19}{221}- \frac{461}{221}\frac{v}{c}}{1+\frac{v}{c}}
\end{equation}

}

\subsection{Theoretische Berechnungen}
\subsection{...}


\end{document}
