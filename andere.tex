\section{Andere Ph\"anomene}

Neben der Pioneer-Anomalie gibt es noch zwei weitere Ph\"anomene, die
den selben Ursprung haben k\"onnten: \foreignlanguage{ngerman}{Die}
Swingby- oder Flyby-Anomalie und das Anwachsen der Astronomischen
Einheit (AU)\cite{Laemmerzahl2006}.

\subsection{Flyby-Anomalie}

In der Raumfahrt l\"asst man Sonden durch ein starkes Gravitationsfeld,
zumeist das eines Planeten, fliegen um sie zu beschleunigen und ihre
Flugbahn zu \"andern. Bei solchen sogenannten Flyby-Man\"overn an der
Erde wurde die Flyby-Anomalie entdeckt. Die Sonden waren nach dem
Vorbeiflug um einige mm/s schneller als berechnet:


\begin{table}[htbn]
\begin{center}
\begin{tabular}{|llcccp{3cm}|}\hline
Mission & Behörde & Datum & Perizentrum $r_p$ & Exzentrizität $e$ & Geschwindigkteits-\newline zuwachs $\Delta v$ \\ \hline
Galileo & NASA & Dez 1990 & 959,9 km & 2,47 & $3,92 \pm 0,08$ mm/s \\
NEAR & NASA & Jan 1998 & 538,8 km & 1,81 & $13,46 \pm 0,13$ mm/s \\
Cassini & NASA & Aug 1999 & 1173 km & 5,8 & 0,11 mm/s \\
Rosetta & ESA & Mär 2005 & 1954 km & 1,327 & $1,82 \pm 0,05$ mm/s \\ \hline
\end{tabular}
\end{center}
\caption{Beobachte Flybys. \label{Table:flyby}}
\end{table}

Die Raumsonde Rosette flog im November 2007 und im November 2009 erneut
an der Erde vorbei. Diese male konnte keine Abweichung von den
berechneten Daten festgestellt werden. Es wird vermutet, dass die Sonde
zu weit von der Erde entfernt war um die Anomalie zu erkennen. Man
vermutet einen gravitativen Effekt hinter der Flyby-Anomalie. Deshalb
erwartet man, dass sich die Auswirkungen bei gr\"o{\ss}erer
Exzentrizit\"at verringern, weil sich dabei die St\"arke und die Dauer
der gravitativen Interaktion verringern. Das selbe wird f\"ur
Exzentrizit\"aten n\"aher bei e=1 erwartet, da \ zum Beispiel bei
Satelliten, welche die Erde umkreisen, keine Anomalien auftreten. F\"ur
das Perizentrum gilt demnach: Je n\"aher es am Zentrum der Gravitation
liegt, desto gr\"o{\ss}er ist die Abweichung. Tr\"agt man also nun
f\"ur die obige Tabelle einmal $\Delta $v als Funktion der
Exzentrizit\"at e und einmal als Funktion des Perizentrums $r_p$ auf, so
erh\"alt man Abbildung \ref{fig:dia}.


\begin{figure}[htb]
\begin{center}
\noindent    
\psfig{figure=images/diagramme,width=\linewidth,height=\textheight,keepaspectratio}
\end{center}
\vskip -10pt
  \caption{{\bf Links:} Die Geschwindigkeitszunahme $\Delta v$ als Funktion der Exzentrizität $e$
{\bf Rechts:} Die Geschwindigkeitszunahme $\Delta v$ als Funktion des Perizentrums $r_p$\cite{Turyshev2004}}
\label{fig:dia}
\end{figure} 


Das gro{\ss}e Problem bei der Anomalie ist, dass zu wenige Daten
vorhanden sind. Es gab bis jetzt nur eine begrenzte Anzahl an
Flyby-Man\"overn und die vorhandenen Daten, sind zu ungenau. 
Meistens taucht die Abweichung zwischen zwei Datenpunkten auf. So ist bis
jetzt nur bekannt um welchen Betrag die Geschwindigkeiten der
verschiedene Sonden von den vorher Berechneten abweichen. Ob es aber
auch Auswirkungen auf die Flugbahnen gab, ist immer noch v\"ollig
unklar. In Zukunft m\"ussen also Flyby-Man\"over genauer untersucht und
vor allem mehr und pr\"azisere Messpunkte aufgenommen werden.

\subsection{Das Anwachsen der Astronomischen Einheit}


Die Astronomische Einheit w\"achst pro Jahr um ca. 10 cm. Dies wurde in
Abstandsmessungen von der Erde zu anderen Planeten entdeckt. Vor allem
die Messungen zum Mars und den ihn umkreisenden Satelliten sind hier
interessant, da sie von 1961 bis 2003 und mit insgesamt 42 Jahren am
l\"angsten liefen. Auch hierf\"ur gibt es bis heute keine ausreichende
Erkl\"arung. Die Auswirkungen der Ausdehnung des Universums sind auf
solchen Skalen n\"amlich viel zu gering.

