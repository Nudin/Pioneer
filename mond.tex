\section{Modifizierte Newtonsche Mechanik (MOND)}

Die modifizierte newtonsche Mechanik (MOND) wurde Mitte der 1980er Jahre
von Mordehai \ Milgrom als Gegenentwurf zur dunklen Materie entwickelt.
Ihre ursprüngliche Motivation ist es die gemessene abflachende
Rotationskurve von Galaxien zu erklären, die sich deutlich von der
Kurve unterscheidet, die nach den keplerschen Gesetzten berechnet
wurde. Das soll erreicht werden indem das zweite newtonsche Gesetz bzw.
das Gravitationsgesetz modifiziert wird. 


\bigskip

Das zweite newtonsche Axiom besagt, dass eine Masse m an der eine Kraft
F anliegt die Beschleunigung a erfährt: 

\begin{equation*}
F=m\cdot a
\end{equation*}
Dieser Zusammenhang kann allerdings für sehr kleine Beschleunigungen nur
sehr schwer bis gar nicht experimentell überprüft und nachgewiesen
werden. In Galaxien bewirkt die Schwerkraft der Sterne aber nur solche
kleinen Beschleunigungen, da sie sehr weit voneinander entfernt sind.
Milgroms Idee \cite{Bekenstein1984} war es daher das zweite newtonsche Gesetz für sehr
kleine Beschleunigungen abzuwandeln in

\begin{equation*}
F=m\cdot a\cdot \mu (a/a_{0})
\end{equation*}
 $a_{0}$ ist eine neue Naturkonstante, die angibt ab welchen
Beschleunigungen die Modifikation wirksam wird. Milgrom bestimmte sie
aus den Messungen der Rotationsgeschwindigkeiten möglichst vieler
Galaxien als ungefähr

\begin{equation*}
a_{0}=2\cdot 10^{-10}\frac{m}{s^{2}}
\end{equation*}
 $\mu (x)$ ist eine unspezifizierte Funktion welche folgende Bedingungen
erfüllt:

\begin{itemize}
\item  $\mu (x\gg 1)\approx 1$, so dass für große Beschleunigungen die
newtonsche Mechanik gilt
\item $\mu (x\ll 1)\approx x$
\end{itemize}
In der Literatur sind für  $\mu (x)$ am häufigsten verwendeten
Funktionen:

{\centering  $\mu (x)=\frac{x}{1+x}$ $\mu
(x)=\frac{x}{\sqrt{1+x^{2}}}$\par}

Will man jetzt MOND in Hinblick auf die Pioneer Anomalie \cite{Turyshev2010}
überprüfen, muss man als erstes betrachten, welche Auswirkungen die
Theorie auf die Zentripetalbeschleunigung  $a_{z}$ einer Masse m hat,
die sich im Gravitationsfeld eines Körpers der Masse M befindet.

Nach Newton gilt mit der Gravitationskonstante G:

{\centering  $F_{g}=\frac{G\cdot {M\cdot m}}{r^{2}}$\par}

Damit gilt für  $a_{z}$

\begin{equation*}
a_{z}=\frac{v^{2}}{r}=\frac{G\cdot {M}}{r^{2}}
\end{equation*}
Löst man diese Gleichung nach  $v^{2}$ auf, so erhält man:

\begin{equation*}
v^{2}=\frac{G\cdot M}{r}
\end{equation*}
Nach MOND gilt für sehr kleine Beschleunigungen:

\begin{equation*}
a_{z}\mu (\frac{a_{z}}{a_{0}})=\frac{G\cdot {M}}{r^{2}}
\end{equation*}
Da  $\frac{a_{z}}{a_{0}}\ll 1$ gilt  $\mu
(\frac{a_{z}}{a_{0}})=\frac{a_{z}}{a_{0}}$

Setzt man dies nun in die obige Gleichung ein und löst nach  $a_{z}$
auf, so erhält man:

\begin{equation*}
a_{z}=\frac{\sqrt{M\cdot G\cdot a_{0}}}{r}
\end{equation*}
Mit  $a_{z}=\frac{v^{2}}{r}$ erhält man hier für  $v^{2}$

\begin{equation*}
v^{2}=\sqrt{M\cdot G\cdot a_{0}}
\end{equation*}
Zusammenfassend gilt also nach der modifizierten newtonschen Dynamik:

\begin{itemize}
\item  $v^{2}=\frac{G\cdot M}{r}$ für  $a_{z}\gg a_{0}$ oder kleine
Abstände
\item  $v^{2}=\sqrt{M\cdot G\cdot a_{0}}$ für  $a_{z}\ll a_{0}$ oder
große Abstände
\end{itemize}
Betrachtet man nun die Pioneer Anomalie, kann man davon ausgehen, dass 
$a_{0}\ll \frac{G\cdot M}{r}$.

Wählt man nun  $\mu (x)=1+\frac{\zeta }{x}$ bekommt man für  $a_{z}$
folgendes Ergebnis:

\begin{equation*}
a_{z}=-\ \frac{G\cdot M}{r^{2}}-\zeta \cdot a_{0}
\end{equation*}
Für  $\zeta =7$ ist 

\begin{equation*}
\zeta \cdot a_{0}=8,4\cdot 10^{8}\frac{\mathit{cm}}{s^{2}}
\end{equation*}
was der anomalen Beschleunigung der Pioneer-Sonden entspricht.