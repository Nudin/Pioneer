
\section{Modifizierte Newtonsche Mechanik (MOND)}


Die modifizierte newtonsche Mechanik (MOND) wurde Mitte der 1980-er
Jahre von Mordehai Milgrom als Gegenentwurf zur dunklen Materie
entwickelt. Ihre urspr\"ungliche Motivation ist es die gemessene
abflachende Rotationskurve von Galaxien zu erkl\"aren, die sich
deutlich von der Kurve unterscheidet, die nach den keplerschen
Gesetzten berechnet wurde. Das soll erreicht werden indem das zweite
newtonsche Gesetz bzw. das Gravitationsgesetz modifiziert wird. 


\bigskip


Das zweite newtonsche Axiom besagt, dass eine Masse m, an der eine Kraft
F anliegt, die Beschleunigung a erf\"ahrt: 

\begin{equation}
F=m\cdot a
\end{equation}

Dieser Zusammenhang kann allerdings f\"ur sehr kleine Beschleunigungen
nur sehr schwer bis gar nicht experimentell \"uberpr\"uft und
nachgewiesen werden. In Galaxien bewirkt die Schwerkraft der Sterne
aber nur solche kleinen Beschleunigungen, da sie sehr weit voneinander
entfernt sind. Milgroms Idee \cite{Bekenstein1984} war es daher das zweite newtonsche
Gesetz f\"ur sehr kleine Beschleunigungen abzuwandeln in

\begin{equation}
F=m\cdot a\cdot \mu (a/a_{0})
\end{equation}

 $a_{0}$ ist eine neue Naturkonstante, die angibt, ab welchen
Beschleunigungen die Modifikation wirksam wird. Milgrom bestimmte sie
aus den Messungen der Rotationsgeschwindigkeiten m\"oglichst vieler
Galaxien als ungef\"ahr

\begin{equation*}
a_{0}=2\cdot 10^{-8}\frac{\mathit{cm}}{s^{2}}
\end{equation*}
$\mu (x)$ ist eine unspezifizierte Funktion, welche folgende
Bedingungen erf\"ullt:

\begin{equation*}
\mu \left( x \right) \approx
\begin{cases}
1 \qquad \mathrm{wenn} \left| x \right| \ll 1 \\
x \qquad \mathrm{wenn} \left| x \right| \gg 1 
\end{cases}
\end{equation*}

In der Literatur sind die f\"ur  $\mu (x)$ am h\"aufigsten verwendeten
Funktionen:

\begin{equation*}
\mu (x)=\frac{x}{1+x}
\qquad
\mu (x)=\frac{x}{\sqrt{1+x^{2}}}
\end{equation*}

Will man jetzt MOND in Hinblick auf die Pioneer Anomalie \cite{Turyshev2010}
\"uberpr\"ufen, muss man als erstes betrachten, welche Auswirkungen die
Theorie auf die Zentripetalbeschleunigung  $a_{z}$ einer Masse m hat,
die sich im Gravitationsfeld eines K\"orpers der Masse M befindet.

Nach Newton gilt mit der Gravitationskonstante G:

\begin{equation}
F_{g}=\frac{G\cdot {M\cdot m}}{r^{2}}
\end{equation}

Damit gilt f\"ur  $a_{z}$

\begin{equation}
a_{z}=\frac{v^{2}}{r}=\frac{G\cdot {M}}{r^{2}}
\end{equation}

L\"ost man diese Gleichung nach  $v^{2}$ auf, so erh\"alt man:

\begin{equation}
v^{2}=\frac{G\cdot M}{r}
\end{equation}

Nach MOND gilt f\"ur sehr kleine Beschleunigungen:

\begin{equation}
a_{z}\mu (\frac{a_{z}}{a_{0}})=\frac{G\cdot {M}}{r^{2}}
\end{equation}

Da  $\frac{a_{z}}{a_{0}}\ll 1$ gilt  $\mu
(\frac{a_{z}}{a_{0}})=\frac{a_{z}}{a_{0}}$

Setzt man dies nun in die obige Gleichung ein und l\"ost nach  $a_{z}$
auf, so erh\"alt man:

\begin{equation}
a_{z}=\frac{\sqrt{M\cdot G\cdot a_{0}}}{r}
\end{equation}

Mit  $a_{z}=\frac{v^{2}}{r}$ erh\"alt man hier f\"ur  $v^{2}$

\begin{equation}
v^{2}=\sqrt{M\cdot G\cdot a_{0}}
\end{equation}

Zusammenfassend gilt also nach der modifizierten newtonschen Dynamik:

\begin{equation*}
v^{2}=
\begin{cases}
\qquad \frac{G\cdot M}{r} \qquad \quad \, \textrm{wenn } a_{z}\gg a_{0} \textrm{ oder kleine Abst\"ande} \\
\sqrt{M\cdot G\cdot a_{0}} \qquad \textrm{wenn } a_{z}\ll a_{0} \textrm{ oder gro{\ss}e Abst\"ande}
\end{cases}
\end{equation*}
Betrachtet man nun die Pioneer-Anomalie, kann man davon
ausgehen, dass $a_{0}\ll \frac{G\cdot M}{r}$. \\
W\"ahlt man nun $\mu (x)=1+\frac{\zeta }{x}$ bekommt
man f\"ur $a_{z}$ folgendes Ergebnis:

\begin{equation}
a_{z}=-\ \frac{G\cdot M}{r^{2}}-\zeta \cdot a_{0}
\end{equation}
F\"ur $\zeta =7$ ist 

\begin{equation*}
\zeta \cdot a_{0}=8,4\cdot 10^{-8}\frac{\mathit{cm}}{s^{2}}
\end{equation*}
was in etwa der anomalen Beschleunigung der Pioneer-Sonden entspricht.


\bigskip


Gundlach et al.\cite{Grundlach2007} nahmen die MOND-Theorie zum Anlass ,das zweite
newtonsche Gesetz f\"ur sehr kleine Beschleunigungen zu \"uberpr\"ufen.
Mit Hilfe eines Torsionspendels gelang es ihnen gute Messungen bis zu
einer Beschleunigung von  $5\cdot 10^{-12}\frac{\mathit{cm}}{s^{2}}$
durchzuf\"uhren. Dabei konnte keine Verletzung von Newtons Aussage
festgestellt werden. Dies stellt allerdings keinen direkten Widerspruch
zur MOND-Hypothese dar, da diese fordert, dass Messungen au{\ss}erhalb
des Einflussbereiches anderer gr\"o{\ss}erer Beschleunigungen
durchgef\"uhrt werden m\"ussen. Diese Bedingung ist auf der Erde durch
ihr Gravitationsfeld und dem des Sonnensystems nicht gegeben. Die
Ergebnisse zeigen jedoch, dass es nicht m\"oglich ist, MOND von
fundamentalen Prinzipien abzuleiten unter der Auflage, dass der
Formalismus  $F=m\cdot a$  unter Laborbedingungen reproduziert.


Nach der modifizierten newtonschen Dynamik m\"ussten auch die Planeten,
vor allem die \"au{\ss}eren, die selbe Anomalie aufweisen wie die
Pioneer Sonden. Die Planetenbahnen sind aber seit langem sehr genau
bekannt und weisen keine Abweichungen auf, die auf eine solche
Beschleunigung hinweisen. Das ist ein weiterer Schwachpunkt der
Theorie.
