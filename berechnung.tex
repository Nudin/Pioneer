
\subsection{Theoretische Berechnungen}
Die ursprüngliche Navigationsberechnung erfolgte durch das Orbital Determination Program (ODP) des JPL.
Hierbei wird die Bahn im Rahmen des relativistischen Einstein-Infeld-Hoffmann-Modells (EIH Modell)
bis einschließlich zur Ordnung $(\frac{v}{c})^4$ berechnet.
Die Gravitation der Sonne, der neun Planeten und des Monds, wurden dabei als isotrope Punktmassen mit dem
PPN-Formalismus (parameterized post-Newtonian formalism) beschrieben\cite{Anderson2002}. Die Gravitation der größten
Asteroiden (ca. 0,2 Erdmassen) und Kometen wurden darüber hinaus gemäß des Newtonschen Gravitationsgesetz mit
ein berechnet. Die Positionen und Massen der Planeten wurden in den früherern Analysen aus der Ephemeride
DE402, später aus DE405 entnommen.\footnote{\textit{``Jet Propulsion Laboratory Development Ephemeris''} sind durch
numerische Integration erzeugte Ephemeriden welche primär für die Raumfahrt gedacht sind.}\cite{Anderson2002}

%Auch die ``Terrestrial and lunar figure effects``,
%die Gezeiten der Erde und die physische Libration wurde mit berücksichtigt.
Die Propagation des Lichtes wurde relativistisch bis zu Ordnung $(\frac{v}{c})^2$ genau berechnet.
%Zu benutzen: Anderson 2002 + Physikjournal + Vorträge
Darüberhinaus wurden bei etlichen dieser Berechnungen eine Reihe von weiteren Einflüssen mit berücksichtigt, welche wir
weiter unten % unter klassische Erklärungen
erläutern werden.\footnote{An dieser Stelle werden oft auch die Einflüsse auf die Beobachtung genannt, welche wir jedoch
bereits im vorhergehenden Abschnitt besprochen haben.}

Natürlich kam schnell die Kritik auf, es könne sich bei der Anomalie um einen Fehler im ODP des JPL handeln.
Die erste Überprüfung erfolgte 1998 mit dem unabhängig entwickelten CHASMP/POEAS-Code des Aerospace
Corporation.
Eine zweite Bestätigung folge 2002 durch einen von C. Markward (Goddard Space Flight Center, GFSC) geschriebenen Code.
Eine weitere Bestätigung erfolgte 2006 durch den Orbit determination code HELIOSAT entwickelt von Ø. Olsen von der
Universität Oslo.
Im Jahr 2008 entwickelte die Groupe Anomalie Pioneer (GAP) eine eigene Software namens ODYSSEY,
mit welcher die Anomalie ebenfalls bestätigt werden konnte.
