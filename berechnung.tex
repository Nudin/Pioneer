
\subsection{Theoretische Berechnungen}
Die ursprüngliche Navigationsberechnung erfolgte durch das Orbital Determination Program (ODP) des JPL.
Hierbei wird die Bahn im Rahmen des relativistischen Einstein-Infeld-Hoffmann-Modells (EIH Modell)
bis einschließlich zur Ordnung $(\frac{v}{c})^4$ berechnet.
Die Gravitation der Sonne, der neun Planeten und des Monds, wurden dabei als isotrope Punktmassen mit dem
PPN-Formalismus (parameterized post-Newtonian formalism) beschrieben\cite{Anderson2002}. Die Gravitation der größten
Asteroiden (ca. 0,2 Erdmassen) und Kometen wurden darüber hinaus gemäß des Newtonschen Gravitationsgesetz mit
ein berechnet.

%Auch die ``Terrestrial and lunar figure effects``, die Gezeiten der Erde und die physische Libration wurde mit
berücksichtigt.
Die Propagation des Lichtes wurde relativistisch bis zu Ordnung $(\frac{v}{c})^2$ genau berechnet.
%Zu benutzen: Anderson 2002 + Physikjournal + Vorträge
Darüberhinaus wurden bei etlichen dieser Berechnungen eine Reihe von weiteren Einflüßen mit berücksichtigt, welche wir
weiter unten % unter klassische Erklärungen
erläutern werden.\footnote{An dieser Stelle werden oft auch die Einflüße auf die Beobachtung genannt, welche wir jedoch
bereits im vorhergehenden Abschnitt besprochen haben.}

Eine unabhängige Überprüfung erfolgte durch durch den CHASMP/POEAS-Code des Aerospace Corporation.
Eine dritte Bestätigung folge später durch einen von C. Markward (Goddard Space Flight Center, GFSC) geschriebenen Code.
Im Jahr 2008 entwickelte Groupe Anomalie Pioneer (GAP) eine eigene Software namens ODYSSEY,
mit welcher die Anomalie ebenfalls bestätigt werden konnte.
Eine weitere Bestätigung erfolgte durch den Orbit determination code der Universität Oslo.