
\subsection{Theoretische Berechnungen der Bahn}
Die ursprüngliche Navigationsberechnung erfolgte durch das Orbital Determination Program (ODP) des JPL.
Hierbei wird die Bahn im Rahmen des relativistischen Einstein-Infeld-Hoffmann-Modells (EIH Modell)
bis einschließlich zur Ordnung $(\frac{v}{c})^4$ berechnet.
Die Gravitation der Sonne, der neun Planeten und des Monds, wurden dabei als isotrope Punktmassen mit dem
PPN-Formalismus (parameterized post-Newtonian formalism) beschrieben\cite{Anderson2002}. Die Gravitation der größten
Asteroiden (ca. 0,2 Erdmassen) und Kometen wurden darüber hinaus gemäß des Newtonschen Gravitationsgesetz mit
ein berechnet. Die Positionen und Massen der Planeten wurden in den früherern Analysen aus der Ephemeride
DE402, später aus DE405 entnommen.\footnote{\textit{``Jet Propulsion Laboratory Development Ephemeris''} sind durch
numerische Integration erzeugte Ephemeriden welche primär für die Raumfahrt gedacht sind.}\cite{Anderson2002}

Die Manöver der Raumsonden wurden als Geschwindigkeitsänderung an bestimmten Zeitpunkten mit einer, durch best-fit Werte,
bestimmten Stärke berücksichtigt.\cite{Levy2008}	% besser

%Auch die ``Terrestrial and lunar figure effects``,
%die Gezeiten der Erde und die physische Libration wurde mit berücksichtigt.
Die Propagation des Lichtes wurde relativistisch bis zu Ordnung $(\frac{v}{c})^2$ genau berechnet. Dies berücksichtigt
vor allem die Shapiro-Verzögerung – ein relativistischer Effekt der besagt, dass sich Licht in der Nähe einer Großen
Masse (in unserem Fall die Sonne, die Planeten und der Mond) für weit entfernte Beobachter langsamer als die
Vakuumlichtgeschwindigkeit zu bewegen scheint. % Erklärung überprüfen/verbessern und Quelle
Die auswirkung der Shapiro-Verzögerung sind jedoch minimal,\cite{Levy2008} so das man diese auch vernachläßigen könnte.
% Solar Corona Effeckt ?
%Zu benutzen: Anderson 2002 + Physikjournal + Vorträge
Darüberhinaus wurden bei etlichen dieser Berechnungen eine Reihe von weiteren Einflüssen mit berücksichtigt, welche wir
weiter unten % unter klassische Erklärungen
erläutern werden.\footnote{An dieser Stelle werden oft auch die Einflüsse auf die Beobachtung genannt, welche wir jedoch
bereits im vorhergehenden Abschnitt besprochen haben.}


Natürlich kam schnell die Kritik auf, es könne sich bei der Anomalie um einen Fehler im ODP des JPL handeln.
Die erste Überprüfung erfolgte 1998 mit dem unabhängig entwickelten CHASMP/POEAS-Code des Aerospace Corporation.
Eine zweite Bestätigung folge 2002 durch einen von C. Markward (Goddard Space Flight Center, GFSC) geschriebenen
Code.
Eine weitere Bestätigung erfolgte 2006 durch den Orbit determination code HELIOSAT entwickelt von Ø. Olsen von der
Universität Oslo.
Im Jahr 2008 entwickelten das Observatoire de la Côte d’Azur (OCA) und Onera, im Auftrag der Groupe Anomalie Pioneer (GAP),
eine eigene Software namens ODYSSEY ("Orbit Determination and phYsical Studies in the Solar Environment Yonder"), zur Analyse der Pioneer-Anomalie.
Dabei achtete man darauf völlig unabhängig und möglichst unterschiedlich zu den Berechnungsverfahren dem ursprünglichen ODP zu sein. 
Auch diese Software bestätigt die Existenz und Größe der Anomalie.\cite{Levy2008}

% Man hat die numerische Genauigkeit der Berechnungen abgeschätzt, sie liegt unter der Fehlergrenze % aus Physikjournal

% wohin damit:
In einem Idealem System würde man alle zu Verfügung stehenden Daten für die Berechnungen verwenden. Jedoch lässt sich nicht verhindern, dass Messdatenpunkte verfälscht werden. Also muss man eine Strategie finden solche Datenpunkte zu finden und auszuschließen oder zu verbessern, ohne dabei selbst die Messung zu verfälschen. Dabei besteht natürlich die Gefahr unwillkürlich Datenpunkte auszuschließen, so dass die Messwerte mit den theoretischen Modellen besser übereinstimmen.
Ausreißer in den Messungen wurden ausgeschlossen, wenn sie im ersten Durchlauf eine Abweichung von mehr als 100 Hz von den erwarteten Wert oder in einem höheren Durchlauf des Algorithmuses eine Abweichung von über $6\sigma$ haten, wobei $\sigma$ die Standardabweichung ist.\cite{Levy2008} % wie hats Anderson gemacht?
% Points with an elevation inferior to 20◦ are rejected so as to limit the effect of imperfections of atmospherical models.
