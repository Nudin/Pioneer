
\subsection{Die Anomalie}
Ab 1979 beobachtet man zwischen der berechneten und der gemessenen Frequenzverschiebung eine anormale Blauverschiebung.
Die Analyse der Daten von 1987 bis 1998 – Entfernungen von 20 bis 70 AU – zeigte eine zeitlich konstante Zunahme dieser
Blauverschiebung von
\begin{equation}
  \frac{d\Delta\nu}{dt}=(5,99\pm0,01)\cdot10^{-9}\frac{Hz}{s}
\end{equation}
wobei $\Delta\nu=[\nu_{Messung}-\nu_{Modell}]'_E$. Diese konstante Zunahme der Blauverschiebung wird meist als eine
Beschleunigung in Richtung Sonne interpretiert.
\begin{equation}
  a_{Pioneer}=\frac{dv}{dt}=-\frac{1}{2}\frac{c}{\nu_E}\frac{d\Delta\nu}{dt} = (8,74\pm1,33)\cdot10^{-8}\frac{cm}{s^2}
\end{equation}
\rem{
oder
\begin{equation}
  \Delta\nu=-\nu_E \frac{2a_p t}{c}
\end{equation}
}
Die genaue Größe der Beschleunigung wird dadurch bestimmt, dass den Simulationen eine unbekannte, konstante Beschleunigung in Richtung Sonne hinzugefügt wird und die Stärke dieser durch einen fit
mit den Messdaten bestimmt wird.	% Quelle, genuaer, korrekt?
Andere Arbeiten mit unterschiedlichen ODPs bestimmten die Beschleunigung zu $(7,70
\pm0,02)\cdot10^{-8}\frac{cm}{s^2}$ (Markwardt,
2002)\footnote{Hier ist jedoch nur der statistische Fehler angegeben}\cite{Markwardt2002} beziehungsweise
$(8,4\pm0,1)\cdot10^{-8}\frac{cm}{s^2}$ (Levy et al., 2008)\cite{Levy2008}.
Wir wollen uns im folgenden jedoch – wie auch praktisch jede Arbeit der Fachliteratur – auf den oben angegebenen von
Anderson et al. berechneten Wert beschränken.
%% wohin damit:
% Die Standartabweichung bei einem Fit mit dieser konstanten Beschleunigung ist deutlich kleiner als ohne. % 9,8 mHz bei Levy

Der Wert mag zwar klein erscheinen, doch ist seine Größenordnung nur das $10^{-5}$ fache der Newtonschen Beschleunigung,
und er ist kleiner als die Faktoren $U/c^2$,$v^2/c^2$,$r a/c^2$ zur relativistischen Korrektur der newtonschen Dynamik.
% richtig?
Seit 1979 ist die Sonde um fast eine halbe Million Kilometer von der Berechneten Bahn abgewichen:
\begin{equation}
  \Delta x= \frac12 \cdot a_p \cdot (2011-1979) a\approx 445.000 km
\end{equation}

Diese Frequenzverschiebung wurde mit nur maximal 3\% Unterschied bei beiden Pioneer Sonden unabhängig von einander
gefunden. Das anomale Signal variiert über den analysierten Zeitrum um nur maximal 3,5\%. Die Richtung der
Beschleunigung ist mit einer Auflösung von 3° bisher noch recht ungenau bestimmt worden. Es ist daher nicht möglich
sicher zu sagen ob die Beschleunigung
in Richtung Sonne, Erde, negativer Geschwindigkeit oder Drehachse geht, dazu später mehr.
