
\subsection{Die Anomalie}
Geht man nun davon aus, dass unsere Physikalische Modelle richtig und wir alle relevanten Einflüsse berücksichtigt haben, so erwartet man für $a_P = 0$ eine Übereinstimmung im fit – im Rahmen der Messungenauigkeit.
Zunächst schien dies auch noch der Fall zu sein, nach dem Flyby-Manöver am Saturn im Jahr 1979 änderte sich dies für Pioneer 11 deutlich. Zu diesem Zeitpunkt befand sich die Sonde in einer Entfernung von etwa 20 AU und somit war die Beschleunigung durch den nur ungenau berechenbaren solaren Strahlungsdruck auf unter $5 \cdot 10^{-8} \frac{cm}{s^2}$ gesunken, %grausammer Satz
somit sank auch die Messungenauigkeit weit genug um das nun auftretenden Phänomen nicht mehr länger zu verschleiern.
Auch für Pioneer 10 stellte man bald darauf eine Abweichung fest.

Die Analyse der Daten von 1987 bis 1998 – das entspricht Sonnen-Entfernungen von 20 bis 70 AU –
zeigte eine zeitlich konstante zunehmende anomale Blauverschiebung von
\begin{equation}
  \frac{d\Delta\nu}{dt}=(5,99\pm0,01)\cdot10^{-9}\frac{Hz}{s}
\end{equation}
wobei $\Delta\nu=[\nu_{Messung}-\nu_{Modell}]'_E$.\cite{Dittus2006} Der Fehler hierbei ist nur der statistische Fehler.

Lässt man, wie oben beschrieben beim fitten eine zusätzliche Beschleunigung zu, so erhält man eine wesentlich bessere Übereinstimmung. Der dabei gefundenen Werte der Anomalie für die Unterschiedlichen Sonden sind:
\begin{eqnarray}
  a_{Pioneer 10} = -(7,84\pm0,01)\cdot10^{-8}\frac{cm}{s^2} \\  
  a_{Pioneer 11} = -(8,55\pm0,02)\cdot10^{-8}\frac{cm}{s^2}
\end{eqnarray}

Zwischen den obigen Werten lässt sich aus Gleichung \ref{equ:rel} ein direkter physikalischer Zusammenhang ableiten. Verwendet man die vereinfachte Version \ref{equ:einf_rel}, so bekommt man:
\begin{equation}
  a_{Pioneer}=\frac{dv}{dt}=-\frac{1}{2}\frac{c}{\nu_E}\frac{d\Delta\nu}{dt}
\end{equation}
%oder
%\begin{equation}
%  \Delta\nu=-\nu_E \frac{2a_p t}{c}
%\end{equation}

Berücksichtigt man den Einfluss aller bekannter Effekte auf den Wert und die Unsicherheit der Größe,\cite{Turyshev2004} so erhält man eine endgültige Größe von:  
\begin{equation}
  a_{Pioneer} = -(8,74\pm1,33)\cdot10^{-8}\frac{cm}{s^2}
\end{equation}

Andere Arbeiten mit den  unterschiedlichen Orbit Determination Codecs bestimmten die Beschleunigung zu $(7,70
\pm0,02)\cdot10^{-8}\frac{cm}{s^2}$ (Markwardt, 2002)\cite{Markwardt2002} beziehungsweise
$(8,4\pm0,1)\cdot10^{-8}\frac{cm}{s^2}$ (Levy et al., 2008)\cite{Levy2008}.
Wobei beide Arbeiten sich nur auf Pioneer 10 beziehen und jeweils nur die statistischen Fehler abgeben.
Wir wollen uns im folgenden jedoch – wie auch praktisch jede Arbeit der Fachliteratur – auf den oben angegebenen von
Anderson et al. berechneten Wert beschränken.
%% wohin damit:
% Die Standartabweichung bei einem Fit mit dieser konstanten Beschleunigung ist deutlich kleiner als ohne. % 9,8 mHz bei Levy

Der Wert mag zwar klein erscheinen, doch ist seine Größenordnung nur das $10^{-5}$ fache der Newtonschen Beschleunigung,
und er ist kleiner als die Faktoren $U/c^2$,$v^2/c^2$,$r a/c^2$ zur relativistischen Korrektur der newtonschen Dynamik.
% richtig?
Seit 1979 ist die Sonde um fast eine halbe Million Kilometer von der Berechneten Bahn abgewichen:
\begin{equation}
  \Delta x= \frac12 \cdot a_p \cdot (2011-1979)^2 a^2\approx 445.000 km
\end{equation}

Diese Frequenzverschiebung wurde mit nur maximal 3\% Unterschied bei beiden Pioneer Sonden unabhängig von einander
gefunden. Das anomale Signal variiert über den analysierten Zeitrum um nur maximal 3,4\%.\cite{Turyshev2004} Die Richtung der
Beschleunigung ist mit einer Auflösung von 3° bisher noch recht ungenau bestimmt worden. Es ist daher nicht möglich
sicher zu sagen ob die Beschleunigung
in Richtung Sonne, Erde, negativer Geschwindigkeit oder Drehachse geht, dazu später mehr.

Eine alternative Interpretation zu einer konstanten Beschleunigung, wäre eine zeitliche Beschleunigung.
So ließe sich die Anomalie auch durch eine Zeitliche Beschleunigung von $a_t = (2,92 \pm 0,044) \cdot 10^{-18} s^{-2}$ schreiben. Zu diesem Ansatz später mehr.
