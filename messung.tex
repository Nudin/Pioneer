
\subsection{Navigation und Geschwindigkeitsmessung}\label{messung}
Die Navigation der Pioneersonden erfolgte mithilfe der Antennen des Deep Space Network (DSN), einem Zusammenschluss mehrerer Radioteleskopanlagen des Jet Propulsion Laboratory (JPL)\footnote{Das Jet Propulsion Laboratory in Kalifornien entwickelt und steuert Sonden für die NASA und beschäftigt viele der Experten auf dem Gebiet der Pioneer-Anomalie, darunter auch John D. Anderson und Slava G. Turyshev.}. Das DSN besteht heute aus großen Radioteleskopanlagen in Goldstone/USA, Madrid/Spanien und Canberra/Australien. Früher gab es darüber hinaus noch Anlagen in Woomera/Australien und Johannesburg/Süd Afrika\cite{Anderson2002}\cite{Turyshev2010}. Dies sind jeweils Komplexe von zahlreichen Antennen – für die Navigation der Pioneer-Sonden wurden laut der Arbeit von Anderson et al.\cite{Anderson2002} davon die Deep Space Station (DSS) Antennen 12, 14, 42, 43, 62 und 63 verwendet. Turyshev und Toth erläutern jedoch in ihrer 2010 erschienenen Arbeit, dass noch etliche weitere Antennen auf alle Parks des DSN, sowie auch einige Antennen anderer Einrichtungen verwendet wurden.\cite{Turyshev2010} Die Antennen hatten Anfangs meist Durchmesser von 26 Metern, später häufig 34 oder 64 Meter, teilweise bis zu 70 Meter\cite{Turyshev2010}.
Man sollte erwähnen, dass diese Antennenkomplexe im Laufe der Zeit vielfach umgebaut wurden um den Anforderungen neuer Missionen gerecht zu werden. Dabei haben sich unter anderem auch die internen Frequenzen geändert\cite{Anderson2002}. Dies muss bei der genauen Betrachtung der Daten berücksichtigt werden, ist darüber hinaus jedoch auch eine Voraussetzung für die 30 Jahre lange Missionsdauer gewesen, da ansonsten die Reichweite der Antennen nur etwa 22 AU \footnote{Eine Astronomische Einheit (AE) oder englisch Astronomical Unit (AU) ist der Abstand zwischen Sonne und Erde $\approx$ 149,6 Millionen Kilometer.} betragen hätte.\cite{Turyshev2010}
Die Geschwindigkeitsmessung der Pioneersonden, welche für die Pioneer-Anomalie von zentraler Bedeutung ist, erfolgte über die Zwei-Wege-Dopplerverschiebung von Radiowellen.

\subsubsection{Entfernungs- und Geschwindigkeitsbestimmung }
Wir nehmen im folgenden an, dass die Sonde sich näherungsweise radial von uns wegbewegt – wir berechnen also genau genommen nur die Geschwindigkeit in Blickrichtung, dies muss bei den Analysen berücksichtigt werden.
Von den Bodenstationen wurden Radiowellen bekannter Frequenz (S-Band, $\sim$2,11 GHz) zum Satelliten gesendet (Uplink).
Die Frequenz wird mithilfe eines Wasserstoff-Masers erzeugt.
Damit werden äußerst präzise und stabile Referenz-Frequenzen von 5 MHz und 10 MHz erzeugt. Im Digital Controlled Oscillator (DCO), werden diese Frequenzen verwendet um mit Frequenzmultipliern ein Signal mit ungefähr 22 MHz zu erzeugten, welches dann mit dem Faktor 96 multipliziert wird um das zu sendende Signal von etwa 2,11 GHz zu erhalten. Nach einer Verstärkung wird das Signal mit einer der Antennen zum Raumsonde gesandt.\cite{Anderson2002}
Der Satellit empfängt das Signal dopplerverschoben:
\begin{equation}
\label{equ:doppler1}
 \nu_R = \frac{1}{\sqrt{1-\frac{v^2}{c^2}}}(1-\frac{v}{c})\nu_E
\end{equation}
Dabei ist $c$ die Lichtgeschwindigkeit, $v$ die Geschwindigkeit der Sonde und $\nu_E$ die Sendefrequenz des Signals auf der Erde und  $\nu_R$ die Frequenz des bei der Raumsonde ankommenden Signals.
Die Sonde antwortet unmittelbar mit einer 8-Watt Sendeanlage (Antennendurchmesser: 137 cm\cite{Markwardt2002}) und eines Transponders
mit einer um den festen (und exakten) Faktor $ \frac{240}{221} $ multiplizierten Frequenz:
\begin{equation}
\label{equ:Faktor}
\nu'_R = \nu_R\frac{240}{221} \approx 2,292 GHz
\end{equation}
Dies ist notwendig, da es sich bei den Radiosignalen um kohärente Wellen handelt und man so Verfälschungen durch Interferenz der hin- und rücklaufenden Wellen vermeidet\cite{Anderson2002}.
Beim Rückweg wird das Signal (Downlink) ein zweites mal identisch dopplerverschoben.
Das empfangene Signal ist also zweifach doppler- und um den Faktor $\frac{240}{221}$ verschoben.
\begin{equation}
 \nu'_E = \frac{1}{\sqrt{1-\frac{v^2}{c^2}}}(1-\frac{v}{c}) \cdot \frac{240}{211}\nu_R \, = \,
\frac{1}{1-\frac{v^2}{c^2}}(1-\frac{v}{c})^2 \cdot \frac{240}{211} \nu_E
\end{equation}
Die relative Verschiebung ergibt sich also zu
\begin{equation}\label{equ:rel}
 \frac{\nu'_E-\nu_E}{\nu_E} = \frac{\frac{19}{221}- \frac{461}{221}\frac{v}{c}}{1+\frac{v}{c}}.
\end{equation}
In einigen Quellen wird zur Veranschaulichung die konstante Frequenzverschiebung durch die Elektronik
vernachlässigt, was zur einfacheren Form führt:
\begin{equation}\label{equ:einf_rel}
 \frac{\nu'_E-\nu_E}{\nu_E} \approx -2\frac{v/c}{1+v/c} \approx -2 \frac{v}{c}
\end{equation}
Ist die Sendeantenne auch die Empfangsantenne, so spricht man von einer zwei-Wege-Messung, wenn Sender- und Empfängerantennen unterschiedlich sind, spricht man von einer drei-Wege-Messung\cite{Levy2009}. Bei den drei-Wege Doppler-Messungen besteht die Gefahr, dass ein unbekannter Zeitunterschied zwischen den Uhren der Antennen die Messung verfälscht. Daher verwendeten manche Analysen diese Daten nicht oder nur selten\cite{Anderson2002}, während andere sie mit zwei-Wege-Doppler-Daten gleich behandelten\cite{Markwardt2002}. Darüber hinaus gibt es eine Vielzahl an sogenannten Einwegs-Doppler-Messdaten, bei denen die Sonde von sich aus die Bodenstation kontaktiert hat. Da die Frequenz der Signalquelle im Weltraumfahrzeug nicht ausreichend genau bekannt ist, sind diese Daten für unsere Zwecke unbrauchbar und werden ignoriert.

Unabhängig davon lässt sich die Entfernung $d$ der Sonde auch durch die Laufzeit $\Delta t$ des Signales bestimmen:
\begin{equation}
 2d = c \Delta t
\end{equation}
Dafür wird der Uplink per Phasenmodulation mit einem Signal versehen und das von der Sonde zurückgesendete Echo beobachtet. (Der Transponder der Sonde demoduliert und filtert es um es in den Downlink hinein zu modulieren.) Dieses Verfahren nennt man "ramping".
Dabei muss man beachten, dass es durch das ständige Senden solcher modulierter Signale und die langen Laufzeiten zu Verwechselungen zwischen unterschiedlichen Signalen kommen kann. Dies muss von den Analyseprogrammen erkannt werden.

Somit hat man zwei voneinander unabhängige Messmethoden, was Konsistenzchecks,
Fehlerminimierung und dem Ausschluss einiger phänomenologischer Fehler ermöglicht. Nicht zuletzt kann man damit durch falsch gemessene Frequenzen verursachte falsche Dopplerdaten erkennen\cite{Anderson2002}.
Allerdings wurden die Laufzeitmessungen laut \cite{Anderson2002} nur bei der Analyse der Daten von Galileo und Ulysses (siehe unten), nicht bei den Pioneersonden verwendet. Von den Pioneersonden liegen Laufzeitmessdaten ("ramped-range") nur von der ersten Zeit der Mission vor. Im späteren Verlauf der Mission wurde diese Technik unbrauchbar, da die Bandbreite der Trägerfrequenz zu klein wurde um die modulierten Veränderungen zu detektieren\cite{Turyshev2010}. % "carrier tracking loop bandwidth"->"Bandbreite der Trägerfrequenz" richtig übersetzt? Erklären warum.

Aufgrund der Eigenrotation der Erde lässt sich außerdem aus der dadurch entstehenden Modulation der Doppler-Daten auch die 3-dimensionale Position der Sonde berechnen. Die Amplitude der Sinusförmigen Variation ist mit dem Deklinationswinkel und die Phase mit der Rektaszension verbunden. Die Position lässt sich dadurch aus einer, einige Tage langen, Reihe von Dopplerdaten bestimmen. Daraus kann man durch Berechnung der Dynamik der Raumsondenbewegung ebenfalls die Entfernung berechnen. Auch dies fließt in die Analysen mit ein.\cite{Anderson2002} % Wirklich die Phase?
Leider ist das S-Band für 3-dimensionale Orbit-Rekonstruktion nur mäßig geignet.\cite{Turyshev2004}

%Laut Markward\cite{Markwardt2002} wurden die Effekte der Dopplerverschiebung auf dem Hinweg bereits beim Senden grob kompensiert, so dass die Frequenz des Signals, das die Sonde empfängt, bei etwa 2.11 GHz liegt.

Die Frequenzmessung erfolgte durch Zählen der Perioden und Vergleich mit einer Atomuhr\cite{Nieto2007}. %``Schwingungen'' als Formulierung überprüfen -  Perioden
Die Frequenz ist dabei einen Durchschnittswert, definiert über die Perioden in einem gewissem Zeitraum, Integrationszeitraum genannt. Die Integrationszeit lag zwischen 0,1 Sekunde und 100 Sekunden oder teilweise noch mehr\cite{Markwardt2002}. Daten mit sehr kurzem Integrationszeitraum nennt man "High rate" Dopplerdaten.

%Zumindest im Zeitraum von 1987-1994 erfolgten die Messungen weitgehend regelmäßig, zusätzlich gab es zu einigen Zeitpunkten eine höhere Anzahl an Messungen\cite{Markwardt2002}. % sonst? ; später weniger?
Laut Markwardt\cite{Markwardt2002} erfolgten die Messungen weitgehend regelmäßig, es gab jedoch zusätzlich zu einigen Zeitpunkten eine höhere Anzahl an Messungen.

%Für die Navigation wurde daraus direkt die aktuelle Flugbahn berechnet, wir wollen uns jedoch im Folgendem auf den – für das Thema relevantere – 
% Vergleich der gemessenen Geschwindigkeit mit der berechneten Geschwindigkeit beschränken. % Oder doch nicht?


\subsubsection{Archivierung der Messdaten}\label{archiv}

Gespeichert wurden die Daten ursprünglich im "Intermediate Data Record"-Format (IDR), nach einer Konversion dann im "Archival Tracking Data File"-Format (ATDF) auf Magnetbändern. Diese enthalten alle vom DSN gemessenen Daten, inklusive Signalstärke, Antennenausrichtung, Frequenz, Entfernung und Störungen\cite{Turyshev2010}. % range und residuals besser übersetzen.
Viele der ATDF Dateien wurden als Magnetbänder an das NSSDC-Archiv zur Archivierung gesandt, jedoch nicht alle – dazu mehr in Kapitel \ref{daten}\cite{Markwardt2002}.
Die Radio Metric Data Conditioning group (RMDC) von JPL's Navigations- und Missionsentwurfs-Abteilung las diese aus und konvertierte sie mit der Software STRIPPER in das Format "Orbit Determination File" (ODF\footnote{Nicht zu verwechseln mit dem verbreiteten Officeformat ODF.} oder ODFILE).
Darin enthalten sind\cite{Levy2008}:
\begin{itemize}
\item Die durchschnittliche Dopplerdrift über eine gewisse Zeitspanne – "Compression time" genannt
\item Die Dauer dieser Zeitspanne
\item Der Zeitpunkt in der Mitte des Intervalls\footnote{Was dem Zeitpunkt entspricht, als die Raumsonde das Signal empfing\cite{Levy2008}}
\item Die Sendefrequenz
\item Angaben dazu, welche DSN-Antennen das Signal geschickt und welche es empfangen haben
\end{itemize}
Auch die Laufzeitmessungen sind in den ODF-Dateien enthalten\cite{Anderson2002}.
Die Compression time beträgt in der Regel 10 s, 60 s, 600 s oder 1980 s\cite{Anderson2002}. % Stimmt das?

Die ODF-Dateien sind das eigentliche Werkzeug mit welchem die meisten an der Pioneer-Anomalie forschenden Teams arbeiten. Diese verwenden die ODF Dateien dann entweder direkt, oder wandeln sie in das Format ihrer Software um (im Fall von \cite{Anderson2002} ist das das NAVIO-Format).

Die Aufzeichnung der Messungen wurde leider nicht so sorgfältig und gründ\-lich durchgeführt, wie man es heute für die Analysen gerne hätte, da man die Anomalie ursprünglich für eine "Kuriosität" hielt\cite{Nieto2005}.
Die Rohdaten wurden von unterschiedlichen Analysten ausgelesen und das oben angesprochene Programm STRIPPER sollte die damals relevanten Navigationsdaten extrahieren\cite{Nieto2005}, wodurch Daten verloren gingen. % merge mit obigen Abschnitt? überprüfen
Dabei verwendeten die unterschiedlichen Analysten unterschiedliche Modelle und Datenbearbeitungsstrategien\cite{Nieto2005}.
Darüber hinaus wurden die Navigations-Daten nicht sorgfältig archiviert\cite{Nieto2005}.
Die Auswirkungen davon werden wir in Kapitel \ref{daten} noch diskutieren.


\subsubsection{Weitere Einflüsse auf die Messung}\label{einfluesse}
Für die genauere Bestimmung der Bahn muss man einige Einflüsse auf die Messung berücksichtigen, welche wir im Folgenden erläutern wollen.

Da das Radiosignal zirkular polarisiert ist, muss bei der Berechnung die Rotation der Sonde berücksichtigt werden: Beim "Reflektieren" des Signals an der Antenne des sich drehenden Raumfahrzeugs kommt es zu einer von der Rotationsgeschwindigkeit abhängigen zusätzlichen Dopplerverschiebung. Jede Umdrehung der Sonde führt zu einer zusätzlichen Schwingung im Up- und im Downlink. Mit dem Frequenzverhältnis von Up- und Downlink ergeben sich insgesamt $(1+240/221)$ Schwingungen pro Umdrehung der Sonde\cite{Anderson2002}. %Ist "Schwingungen" das richtige Wort dafür
Gleichung \ref{equ:Faktor} muss wie folgt erweitert werden:
\begin{equation}
\nu'_R = \frac{240}{221}\nu_R - \eta\nu_{Spin} \quad \mathrm{mit}  \quad \eta = 1+ \frac{240}{221}\ .
\end{equation}
Die durchschnittliche Rotationsgeschwindigkeit liegt bei etwa 4,4 Umdrehungen pro Minute (rpm) für Pioneer 10 und etwa 7,25 rpm für Pioneer 11 und sank mit der Zeit\cite{Anderson2002}.
Hochqualitative Daten zum Eigenrotation (Spin) sind für Pioneer 10 nur bis zum 17. Juli 1990 verfügbar, als das DSN aufhörte Spinkalibrationen durchzuführen. Für spätere Daten muss der Spin des Raumschiffes durch Interpolation der Datenpunkte und den Daten des Imaging Photo Polarimeter (IPP) berechnet werden. Nach einem Manöver am 6. Juli 1993 reichte die Energie jedoch für dessen Betrieb nicht mehr aus. Analysten konnten jedoch noch etwa alle 6 Monate eine grobe Abschätzung des Spins aus Informationen der sogenannten ConScan-Manöver erhalten. % Conscan erklären
Bei ConScan – Kurzform für conical scan, auf deutsch auch Minimumpeilung – wird die Empfangsantenne kreisförmig bewegt und gemessen, wo das Signal am stärksten ist. Führt man dieses Verfahren mehrfach durch, kann man die ideale Ausrichtung der Antenne herausfinden. In Kombination mit einem Ausrichtungsmanöver des Raumfahrzeugs kann man damit auch die ideale Ausrichtung der Antenne der Sonde bestimmen.\cite{Anderson2002}
%aus Anderson2002:
%Conscan stands for conical scan. The receiving antenna
%is moved in circles of angular size corresponding to one
%half of the beam-width of the incoming signal. This pro-
%cedure, possibly iterated, allows the correct pointing di-
%rection of the antenna to be found. When coupled with
%a maneuver, it can also be used to find the correct point-
%ing direction for the spacecraft antenna. The precession
%maneuvers can be open-loop, for orientation towards or
%away from Earth-pointing, or closed-loop, for homing on
%the uplink radio-frequency transmission from the Earth.
Für die Daten nach 1995 wurde der Spin nicht mehr berechnet, auch wenn dies weiterhin mit den Aufgezeichneten ConScan-Daten möglich wäre. Außerdem ist die Spinachse nicht genau identisch mit der Phasenachse weshalb es eine sehr kleine, aber messbare, Sinusfunktion in den Dopplerdaten gibt. Daraus ließe sich ebenfalls die Spinrate für die nach 1993 gewonnen Daten berechnen – dies wurde bisher jedoch noch nicht gemacht.\footnote{Zumindest soweit uns bekannt.}
Die genaue Spinkalibrierung von Pioneer 11 ist aufgrund des Versagens eines Spin-down-Schubtriebwerks nicht möglich\cite{Anderson2002}.
Markwardt zeigte in seiner Analyse, dass für Pioneer 10 der Spin vernachlässigt oder durch den durchschnittlichen Spin von 4,4 rpm vereinfacht werden kann, ohne dass sich das Ergebnis nennenswert ändert\cite{Markwardt2002}.

Zu beachten ist, dass die Propagation des Signales vom Medium beeinflusst wird. Der Einfluss von interplanetarer Materie konnte durch einen Vergleich
der Daten mit denen der Cassini-Mission analysiert werden, da diese mehrere Frequenzbänder verwendete\cite{Dittus2006}. %mehr, richitg, quelle 
Der Einfluss der Ionosphäre und der Troposphäre auf das Singal wurde durch Implementation der International Reference Ionosphere (IRI)
beziehungsweiße der Global Mapping Functions (GMF) berücksichtigt\cite{Levy2008}. % Anderson auch?; ist das Propagation?; Erklären?

%Gerechnet wurde in der Standard-Epoche J2000.0. %mehr, wohin damit?

Die Bewegung der Sonde wurde in baryzentrischen Koordinaten gemäß ICRF beschrieben. % genauer
Da die gemessene Geschwindigkeit jedoch die Relativgeschwindigkeit zur auf der Erde stehenden Antenne ist,
muss man den Einfluss dieser Geschwindigkeit berechnen um die Geschwindigkeit und somit die Sondenbahn im baryzentrischen Koordinatensystem zu erhalten.
Anders gesagt enthält die Frequenzverschiebung noch einen weiteren Term bezüglich der Bewegung der Antenne. Aus Gleichung \ref{equ:doppler1} wird somit:
\footnote{Markwardt gibt in seiner entsprechenden Formel beide Male ein $c^2$ anstatt c an, wir vermuten dass es sich dabei nur um einen Tippfehler in der Arbeit handelt.}
\begin{equation}
\label{equ:doppler3}
 \nu_R = \frac{(1-\frac{v}{c})}{\sqrt{1-\frac{v^2}{c^2}}} \cdot \frac{\sqrt{1-\frac{v_E^2}{c^2}}}{(1-\frac{v_E}{c})} \cdot \nu_E 
\end{equation}
Zusammen mit den Effekten des Spins ergibt sich damit die gesamte Frequenzverschiebung in Abhängigkeit von der Sondengeschwindigkeit in baryzentrischen Koordinaten zu:
\begin{equation}
 \nu'_E = \left[ \frac{240}{221} \nu_E d_{\overline{ER}} - \eta \nu_{Spin} \right]  d_{\overline{RE}}
\end{equation}
Wobei $d_{\overline{ER}}$ und $d_{\overline{RE}}$ die Dopplerverschiebungsterme gemäß Gleichung \ref{equ:doppler3} sind.
Die Erde ist jedoch sehr dynamisch: um die Präzision, die für diese Belange gewünscht ist, zu erreichen, muss die Geschwindigkeit der Antennen auf der Erdoberfläche unter der Berücksichtigung von Prezission, Nutation,
siderischen Rotation, Polarbewegung, der Gezeitenkräfte und plattentektonischen Bewegungen bestimmt werden.
Die Angaben zu Abbremsung sowie Unregelmäßigkeit der Rotation, zur Polbewegung, die Loveschen Zahlen\footnote{Von A. E. H. Love beschriebene ``Proportionalitätsfaktoren zwischen den verschiedenen Verzerrungen sowie dem sich einstellenden Gravitationsfeld einer sphärisch symmetrischen, nichtrotierenden elastischen isotropen Kugel und einem äußerem an der Kugel angreifenden Grafitationsgradienten''\cite{Dittus2006}.} und der Chandler wobble\footnote{Spiralfomiges Schwingen
der Erdachse um 0,7 Bogensekunden mit einer Periode von 435 Tagen} % Begriffe (genauer) erklären?
wurden dabei direkt aus Messungen mit Lunar Laser Ranging (LLR)\footnote{Beim LLR wird die Laufzeiten von Laserpulsen gemessen, welche von Spiegeln auf dem Mond reflektierten werden.},
Satellite Laser Ranging (SLR)\footnote{Beim SLR wird die Laufzeit von Laserpulsen zwischen einem Satellit und der Bodenstation gemessen.} und Very Long Baseline Interferometry
(VLBI)\footnote{Beim VLBI-Verfahren\cite{vlbi}, wird das selbe Radiosignal
von mehreren, weit auseinander stehenden Antennen empfangen. Die Daten
von den einzelnen Antennen werden zusammen mit einer einheitlichen
Zeitinformation gespeichert. Sp\"ater werden die Daten und
zugeh\"origen Zeiten der verschiedenen Antennen verglichen. So erh\"alt
man den den Laufzeitunterschied des Signals zu den Antennen und kann
daraus die Entfernung des Senders bestimmen. Da die Antennen, wegen der
Speicherung der Daten, nicht mit einem Kabel verbunden sein m\"ussen,
k\"onnen sie sehr weit weg voneinander aufgestellt werden, was das
Aufl\"osungsverm\"ogen des Verfahrens deutlich erh\"oht.} bestimmt.
Diese Daten wurden früher von Publikationen der International Earth Rotation Service (IERS) und der United States Naval Observatory (USNO) zusammengetragen. Heute werden die Daten vom ICRF bereitgestellt,
zu welchen die Earth Orientation Parameters (EOP) des JPL viel beiträgt\cite{Anderson2002}.

Für die Analysen wurden etliche unterschiedliche Zeiten verwendet. So enthalten die Ephemeriden Zeitangaben in der Ephemeridenzeit (ET),
welche für die Berechnungen in die Internationale Atomzeit (TAI) umgerechnet wurden. Die ATDF-Dateien enthalten die Zeitangaben in der koordinierten Weltzeit (UTC),
die ODF-Dateien jedoch in Universal Time 1 (UT1)\footnote{Durch astronomische Beobachtung gewonnene und um Einflüsse der Polschwankungen (mit Perioden über 7 Tage) korrigierte, mittlere Ortszeit des durch die Sternwarte von Greenwich führenden Nullmeridians}. Die Zeiten mussten für die Analysen, unter Berücktischtigung der Erdbewegung, umgerechnet werden.
Auch hierfür verwendete man äußerst genaue Angaben zu Position, Geschwindigkeit und Gravitationspotential der Antennen.\cite{Dittus2006}\cite{Anderson2002}\cite{Markwardt2002}

An dieser Stelle sei erwähnt, das die beiden vorangegangenen Abschnitte sich primär auf die Berechnungen von Anderson et al. aus dem Jahr 2002 beziehen.
Wir werden im nächsten Kapitel sehen, dass es mehrere unabhängige Überprüfungen gegeben hat.
Diese verwenden zum Teil andere Koordinaten- und Zeitsysteme.
So rechnet das Programm ODESSEY mit der Barycentric Coordinate Time (TCB) im Barycentric Celestial Reference System (BCRS).
Dies spielt jedoch für die Betrachtung im Rahmen dieser Arbeit keine Rolle, da die Unterschiede gering sind und die meisten fortführenden Arbeiten auf die Berechnungen von Anderson et al. aufbauen.	%fortführende Arbeiten?

Man hat sogar einen möglichen Einfluss von mechanischer Deformation der Antennen der Bodenstationen durch ihr eigenes Gewicht,
Alterung, Wind, Tektonik, etc. abgeschätzt, und in die Fehlerrechnung mit einbezogen\cite{Dittus2006}. % mehr, Quelle (note: <10^-5 a_p)

% Es fehlt u.U. noch einige weitere Dinge die man hier erwähnen sollte

% (wo) kommt das rein:
% Da die im ODF format gespeicherten Messdaten nur die Zeit t2 (Signal an der Sonde) nicht jedoch
% die Zeitpunkte t1 (Sendezeitpunkt auf der Erde) und t3 (Empfangszeitpunkt auf der Erde) enthalten,
% müssen dise für die spätere Verwendung in Analysen durch Schritweiße Berechnung der
% "relativistic light time equations"" aus t2 errechnen.\cite{Levey2009} %FIXME: begriff übersetzen

