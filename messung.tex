
\subsection{Navigation und Geschwindigkeitsmessung}
Die Navigation der Pioneersonden erfolgte mithilfe der Antennen des Deep Space Network (DSN) einem Zusammenschluss mehrerer Radioteleskopanlagen des Jet Propulsion Laboratory (JPL)\footnote{Das Jet Propulsion Laboratory in Kalifornien entwickelt und steuert Sonden für die NASA und beschäftigt viele der Experten auf dem Gebiet der Pioneeranomalie, darunter auch John D. Anderson und Slava G. Turyshev}. Das DSN besteht heute aus großen Radioteleskopanlagen in Goldstone/USA, Madrid/Spanien und Canberra/Australien. Früher gab es darüber hinaus noch Anlagen in Woomera/Australien und Johannesburg/Süd Afrika.\cite{Anderson2002}\cite{Turyshev2010} Dies sind jeweils Komplexe von zahlreichen Antennen – für die Navigation der Pioneer-Sonden wurden laut der Arbeit von Anderson et al., 2002\cite{Anderson2002} davon die Deep Space Station (DSS) Antennen 12, 14, 42, 43, 62 und 63 verwendet. Turyshev und Toth erläutern jedoch in ihrer 2010 erschienenen Arbeit, dass noch etliche weitere Antennen auf alle Parks des DSN, sowie auch einige Antennen anderer Einrichtungen verwendet wurden.\cite{Turyshev2010} Die Antennen hatten Anfangs meist Durchmesser von 26 Metern, später häufig 34 oder 64 Meter teilweise bis zu 70 Meter.\cite{Turyshev2010}
Man sollte erwähnen, dass diese Antennenkomplexe im laufe der Zeit zahlreich umgebaut wurden um den Anforderungen neuer Missionen gerecht zu werden. Dabei haben sich unter anderem auch die internen Frequenzen geändert\cite{Anderson2002}. Dies muss bei der genauen Betrachtung der Daten berücksichtigt werden, ist darüber hinaus jedoch auch eine Voraussetzung für die 30 Jahre lange Missionsdauer gewesen, da ansonsten die Reichweite der Antennen nur bei etwa 22 AU gelegen hätte.\cite{Turyshev2010}
Die Geschwindigkeitsmessung der Pioneersonden, welche für die Pioneeranomalie von zentraler Bedeutung ist, erfolgte über die Zwei-Wege-Dopplerverschiebung von Radiowellen:

Wir nehmen im folgenden an, dass die Sonde sich näherungsweise radial von uns wegbewegt.
Von den Bodenstationen wurden Radiowellen bekannter Frequenz (S-Band, $\sim$2,11 GHz) zum Satelliten gesendet (uplink).
Die Frequenz wird mithilfe eines Wasserstoff-Masers erzeugt.
Damit wird eine äußerst präzise und stabile Referenz-Frequenz von 5 MHz und 10 MHz erzeugt. Im Digital Controlled Oscillator (DCO), werden diese Frequenz verwendet um mit frequenzmultipliern ein Signal mit ungefähr 22 MHz zu erzeugten, welches dann mit dem Faktor 96 multipliziert wird um das zu sendende Signal von etwa 2,11 GHz zu erhalten.\cite{Anderson2002}
Der Satellit empfängt das Signal dopplerverschoben:
\begin{equation}
 \nu_R = \frac{1}{\sqrt{1-\frac{v^2}{c^2}}}(1-\frac{v}{c})\nu_E
\end{equation}
und antwortet unmittelbar mittels einer 8-Watt Sendeanlage und eines Transponders
mit einer um den festen Faktor $ \frac{240}{221} $ multiplizierten Frequenz:
\begin{equation}
\nu'_R = \nu_R\frac{240}{221}
\end{equation}
Dies ist notwendig, da es sich bei den Radiosignalen um kohärente Wellen handelt und man so Verfälschungen durch Interferenz der Hin- und Rücklaufenden Wellen vermeidet.\cite{Anderson2002}
Beim Rückweg wird das Signal ein zweites mal identisch dopplerverschoben.
Das empfangene Signal ist also zweifach dopplerverschoben und um den Faktor $\frac{240}{221}$ verschoben.
\begin{equation}
 \nu'_E = \frac{1}{\sqrt{1-\frac{v^2}{c^2}}}(1-\frac{v}{c}) \cdot \frac{240}{211}\nu_R \, = \,
\frac{1}{1-\frac{v^2}{c^2}}(1-\frac{v}{c})^2 \cdot \frac{240}{211} \nu_E
\end{equation}
Die relative Verschiebung ergibt sich also zu
\begin{equation}
 \frac{\nu'_E-\nu_E}{\nu_E} = \frac{\frac{19}{221}- \frac{461}{221}\frac{v}{c}}{1+\frac{v}{c}}.
\end{equation}
In vielen Quellen wird die konstante Frequenzverschiebung durch die Elektronik der Quelle vernachlässigt, was zur
einfacheren Form von
\begin{equation}
 \frac{\nu'_E-\nu_E}{\nu_E} \approx -2\frac{v/c}{1+v/c} \approx -2 \frac{v}{c}
\end{equation}
führt.
Sind Sender- und Empfängerantenne die selben, so spricht man von einer zwei-Wege-Messung, wenn Sende und Empfängerantennen unterschiedlich sind spricht man von einer drei-Wege-Messung.\cite{Levy2009} Bei den drei-Wege Doppler-Messungen besteht die Gefahr, dass ein unbekannter Zeitunterschied zwischen den Antennen die Messung verfälscht, daher verwendete man diese Daten meist nicht.\cite{Anderson2002} %schon oben, als footnote?
Darüber hinaus lässt sich die Entfernung $d$ der Sonde auch durch die Laufzeit $\Delta t$ des Signales bestimmen:
\begin{equation}
 2d = c \Delta t
\end{equation}
Dafür wird der Uplink per Phasenmodulation mit einem Signal versehen und das von der Sonde zurückgesendete Echo beobachtet. (Der Transponder der Sonde demoduliert und filtert es um es beim Down-link wieder hinein zu modulieren.)
Dabei muss man beachtet, das durch das ständige senden solcher Modulierten Signale und die langen Laufzeiten es zu Verwelchslungsgefahr zwischen unterschiedlichen Signalen kommen kann. Dies muss von den Analyseprogrammen erkannt werden.

Somit hat man zwei voneinander unabhängige Messmethoden, was Konsitenzchecks,
Fehlerminimierung und Ausschluss einiger phänomenologischer Fehler ermöglicht. Nicht zuletzt kann man damit durch Frequenzfehler kommende falsche Dopplerdaten erkennen.\cite{Anderson2002}
Allerdings wurde dies laut \cite{Anderson2002} nur bei der Analyse der Daten von Galileo und Ulysses (siehe unten), nicht bei den Pioneersonden verwendet.

Aufgrund der Eigenrotation der Erde lässt sich außerdem aus der dadurch entstehenden Modulation der Doppler-Daten auch die 3 dimensionale Position der Sonde berechnen. Die Amplitude der Sinusförmigen Variation ist mit dem Deklinationswinkel und die Phase mit der Rektaszension verbunden. Die Position lässt sich dadurch aus einem Satz Dopplerdaten gewinnen, voraussetzt dieser ist einige Tage lang. Auch daraus kann man durch Berechnung der Dynamik der Raumsondenbewegung die Entfernung berechnen.Auch dies fließt in die Analysen mit ein.\cite{Anderson2002} % Wirklich die Phase?
%Aber nich besonders gut, wie wir später sehen werden!?	 	% mehr

Die Frequenzmessung erfolgte durch Zählen der Perioden und Vergleich mit einer Atomuhr.\cite{Nieto2007} %``Schwingungen'' als Formulierung überprüfen -  Perioden
Die Frequenz ist die eine durchschnittliche Frequenz, definiert über die Perioden über einen gewissen Zeitraum.

Für die Navigation wurde daraus direkt die aktuelle Flugbahn berechnet, wir wollen uns jedoch im Folgendem auf den – für das Thema relevantere – 
Vergleich der gemessenen Geschwindigkeit mit der berechneten Geschwindigkeit beschränken. % Oder doch nicht?

Für die genauere Bestimmung der Bahn muss man einige weitere Einflüsse berücksichtigen, welche wir im folgenden erläutern wollen.

Da das Radiosignal zirkular polarisiert ist, muss bei der Berechnung die Rotation der Sonde berücksichtigt werden: Beim "reflektieren" des Signals an der Antenne des drehenden Raumschiffes kommt es zu einer von der Rotationsgeschwindigkeit abhängigen Dopplerverschiebung. Jede Umdrehung der Sonde führt zu einer zusätzlichen Schwingung im Up- und im Down-link. Mit dem Frequenz-Verhältnis von up- und Down-link ergibt sich insgesamt $(1+240/221)$ Schwingungen pro Umdrehung der Sonde.\cite{Anderson2002} %Ist "Schwingungen" das richtige Wort dafür
Hochqualitative Daten zum Spin sind für Pioneer 10 nur bis zum 17. Juli 1990 verfügbar, als das DSN aufhörte Spinkalibrationen durchzuführen. Für spätere Daten muss der Spin des Raumschiffes durch Interpolation der Datenpunkte und den Daten des Imaging Photo Polarimeter (IPP) berechnet werden. Nach dem Manöver am 6 July 1993 reiche die Energie jedoch nicht mehr für den Betrieb dieses aus. Analysten konnten jedoch noch etwa alle 6 Monate eine grobe Abschätzung des Spins aus Informationen der Conscan-Manöver erhalten. % Conscan erklären
%aus Anderson2002:
%Conscan stands for conical scan. The receiving antenna
%is moved in circles of angular size corresponding to one
%half of the beam-width of the incoming signal. This pro-
%cedure, possibly iterated, allows the correct pointing di-
%rection of the antenna to be found. When coupled with
%a maneuver, it can also be used to find the correct point-
%ing direction for the spacecraft antenna. The precession
%maneuvers can be open-loop, for orientation towards or
%away from Earth-pointing, or closed-loop, for homing on
%the uplink radio-frequency transmission from the Earth.
Für die Daten nach 1995 wurden der Spin nicht mehr berechnet, auch wenn dies weiterhin mit den Aufgezeichneten conscan Daten möglich wäre. Außerdem, ist die Spinachse nicht genau identisch mit der Phasenachse, weshalb es eine sehr kleine – aber messbare – Sinusfunktion in den Dopplerdaten gibt. Daraus ließe sich ebenfalls die Spinrate für die Daten von nach 1993 berechnen – dies wurde bisher jedoch noch nicht gemacht.\footnote{Zumindest soweit uns bekannt.}
Die genaue Spinkalibration von Pioneer 11 ist aufgrund des Versagens eines Spin-down-Schubtriebwerks nicht möglich.\cite{Anderson2002}

Zu beachten ist, dass die Propagation des Signales vom Medium beeinflusst wird. Der Einfluss von interplanetarer Materie konnte durch Vergleich
der Daten mit denen der Cassini-Mission analysiert werden, da diese mehere Frequenzbänder verwendet.\cite{Dittus2006} %mehr, richitg, quelle 
Der Einfluss der Ionosphäre und der Troposphäre auf das Singal wurde durch Implementation der International Reference Ionosphere (IRI)
beziehungsweiße der Global Mapping Functions (GMF) berücksichtigt.\cite{Levy2008} % Anderson auch?; ist das Propagation?; Erklären?

Gerechnet wurde in der Standard-Epoche J200.0. %mehr

Die Bewegung der wurde in baryzentrischen Koordinaten gemäß ICRF beschrieben, % genauer
da die Erde jedoch sehr dynamisch ist muss – für eine hinreichend genaue Bestimmung der Sondenbahn –
die Bewegung der Antennen auf der Erdoberfläche unter der Berücksichtigung von Prezission, Nutation,
siderischen Rotation, Polarbewegung, der Gezeitenkräfte und Platentektonischen Bewegungen berücksichtigt werden.
Die Angaben zu Abbremsung, sowie zur Unregelmäßigkeit der Rotation, zur Polbewegung, die Love numbers\footnote{Von AEH Love beschriebene ``Proportionalitätsfaktoren zwischen den verschiedenen Verzerrungen sowie dem sich einstellenden Gravitationsfeld einer sphärisch symmetrischen, nichtrotierenden elastischen isotropen Kugel und einem äußerem an der Kugel angreifenden Grafitationsgradienten''\cite{Dittus2006}} und der Chandler wobble\footnote{Spiralfomiges Schwingen
der Erdachse mit einer Periode von 435 Tagen} % Begriffe (genauer) erklären? 0,7 Bogensekunden
wurden dabei direkt aus Messungen mit Lunar Laser Ranging (LLR)\footnote{Beim LLR wird die Laufzeit von am Mond von Spiegeln reflektierten Laserpulsen gemessen},
Satellite Laser Ranging (SLR)\footnote{Beim SLR wird die Laufzeit von Laserpulsen zwischen einem Satellit und der Bodenstation gemessen} und Very Long Baseline Interferometry
(VLBI)\footnote{Dabei werden von zwei, interkontinental weit entfernten Radioteleskopen, die Signale inklusive
Zeitreferenz gespeichert und die Interferenz dieser Signale am Computer stimuliert} bestimmt.
Diese Daten wurden früher von Publikationen der International Earth Rotation Service (IERS) und der United States Naval Observatory (USNO) zusammengetragen. Heute werden die Daten vom ICRF bereitgestellt, zu welchen die Earth Orientation Parameters
(EOP) des JPL eine große Rolle beitragen.\cite{Anderson2002}

Die Beobachtungen der Bodenstationen wurden mit Zeitangaben nach der Universal Time 1 (UT1)
\footnote{Durch astronomische Beobachtung gewonnene und um Einflüsse der Polschwankungen (mit Perioden über 7 Tage) korrigierte, mittlere Ortszeit des durch die Sternwarte von Greenwich führenden Nullmeridians}
parametrisiert gespeichert. Für die Analysen und
Berechnungen musste diese Zeitangaben einerseits in die Internationale Atomzeit TAI, andererseits auch in die
Ephemeridenzeit umgerechnet werden. Auch hierfür verwendete man äußerst genaue Angaben zu Position, Geschwindigkeit und
Gravitationspotential der Antennen.\cite{Dittus2006}

An dieser Stelle sei erwähnt, das die beiden vorangegangenen Abschnitte sich auf die Berechnungen von Anderson et al. auf dem Jahr 2002 beziehen.
Wir werden im Nächsten Kapitel sehen, dass es merdere unabhängige Überprüfungen gegeben hat.
Diese verwenden zum Teil andere Koordinaten- und Zeitsysteme.
So rechnet das Progeramm ODESSEY mit der Barycentric Coordinate Time (TCB) im Barycentric Celestial Reference System (BCRS).
Dies spielt jedoch für die betrachtung im Rahmen dieser Arbeit keine Rolle, da die unterschiede gering sind und die meisten fortführenden Arbeiten auf die Berechnngen von Anderson et al. aufsetzen.	%fortführende Arbeiten?

Mann hat sogar einen möglichen Einfluss von mechanischer Deformation der Antennen der Bodenstationen durch ihr eigenes Gewicht,
Alterung, Wind, Tektonik, etc. abgeschätzt, und in die Fehlerrechnung mit einbezogen.\cite{Dittus2006} % mehr, Quelle (note: <10^-5 a_p)

% Es fehlt u.U. noch einige weitere Dinge die man hier erwähnen sollte.

Die Aufzeichnung der Messungen wurde leider nicht so sorgfältig und gründlich durchgeführt, wie man heute für die Analysen gerne hätte, da man die Anomalie ursprünglich für eine "Kuriosität" hielt.\cite{Nieto2005} Die Rohdaten wurden von unterschiedlichen Analysten ausgelesen und mit dem Programm STRIPPER/VAX behandelt um die damals relevanten Navigationsdaten zu extrahieren.\cite{Nieto2005} % 
Dabei verwendeten die unterschiedlichen Analysten unterschiedliche Modelle und Datenbearbeitungsstrategien.\cite{Nieto2005}
Darüber hinaus würden die Navigations-Daten nicht sorgfältig archiviert.\cite{Nieto2005}

Gespeichert wurden die Daten ursprünglich im "Intermediate Data Record (IDR)"-Format, nach einer Konversion dann im "Archival Tracking Data File"-Format (ATDF) auf Magnetbändern.\cite{Turyshev2010} das Team um Anderson las diese aus und konvertierte sie mit Standartsoftware in das Format "Orbit Determination File" (ODF\footnote{Nicht zu verwechseln mit dem verbreiteten Officeformat ODF.} oder ODFILE), dabei wird darin der durchschnittliche Dopplerdrift über eine gewisse Zeitspanne
– "Compression time" genannt – sowie die dauer dieser Zeitspanne, der Zeitpunkt in der Mitte des Intervals\footnote{Was dem Zeitpunkt entspricht, als die Raumsonde das Signal empfing\cite{Levy2008}}, die Sendefrequenz und
Angaben dazu welche DSN Antennen das Signal geschickt und welche es empfangen haben.\cite{Levy2008}
Auch die Laufzeitmessungen sind in den ODF-Dateien enthalten.\cite{Anderson2002}
Mit diesen Dateien arbeiten alle im folgenden beschriebenen Programme.
Die Copression time beträgt 10 s, 60 s, 600 s oder 1980 s.\cite{Anderson2002} % Stimmt das?

% (wo) kommt das rein:
% Da die im ODF format gespeicherten Messdaten nur die Zeit t2 (Signal an der Sonde) nicht jedoch
% die Zeitpunkte t1 (Sendezeitpunkt auf der Erde) und t3 (Empfangszeitpunkt auf der Erde) enthalten,
% müssen dise für die spätere Verwendung in Analysen durch Schritweiße Berechnung der
% "relativistic light time equations"" aus t2 errechnen.\cite{Levey2009} %FIXME: begriff übersetzen


