
\subsection{Navigation und Geschwindigkeitsmessung}
Die Navigation der Pioneersonden erfolgte mithilfe der Antennen des Deep Space Network (DSN) in Goldstone/USA,
Madrid/Spanien und Canberra/Australien.
Die Geschwindigkeitsmessung der Pioneersonden, welche für die Pioneeranomalie von zentraler Bedeutung ist, erfolgte über
die Zwei-Wege-Dopplerverschiebung von Radiowellen:

Wir nehmen im folgenden an, dass die Sonde sich näherungsweise radial von uns wegbewegt.
Von den Bodenstationen wurden Radiowellen bekannter Frequenz (S-Band, ~2,11 GHz) zum Satelliten gesendet (uplink).
Der Satellit empfängt das Signal dopplerverschoben:
\begin{equation}
 \nu_R = \frac{1}{\sqrt{1-\frac{v^2}{c^2}}}(1-\frac{v}{c})\nu_E
\end{equation}
und antwortet unmittelbar mittels einer 8-Watt Sendeanlage und eines Transponders
mit einer um den festen Faktor $ \frac{240}{221} $ verschobenen Frequenz:
\begin{equation}
\nu'_R = \nu_R\frac{240}{211}
\end{equation}
Beim Rückweg wird das Signal ein zweites mal identisch dopplerverschoben.
Das empfangene Signal ist also zweifach dopplerverschoben und um den Faktor $\frac{240}{221}$ verschoben.
\begin{equation}
 \nu'_E = \frac{1}{\sqrt{1-\frac{v^2}{c^2}}}(1-\frac{v}{c}) \cdot \frac{240}{211}\nu_R \, = \,
\frac{1}{1-\frac{v^2}{c^2}}(1-\frac{v}{c})^2 \cdot \frac{240}{211} \nu_E
\end{equation}
Die relative Verschiebung ergibt sich also zu
\begin{equation}
 \frac{\nu'_E-\nu_E}{\nu_E} = \frac{\frac{19}{221}- \frac{461}{221}\frac{v}{c}}{1+\frac{v}{c}}.
\end{equation}
In vielen Quellen wird die konstante Frequenzverschiebung durch die Elektronik der Quelle vernachlässigt, was zur
einfacheren Form von
\begin{equation}
 \frac{\nu'_E-\nu_E}{\nu_E} \approx -2\frac{v/c}{1+v/c} \approx -2 \frac{v}{c}
\end{equation}
führt.
Darüber hinaus lässt sich die Entfernung $d$ der Sonde auch durch die Laufzeit $\Delta t$ des Signales bestimmen:
\begin{equation}
 2d = c \Delta t
\end{equation}
Somit hat man zwei voneinander unabhängige Messmethoden, was Konsitenzchecks,
Fehlerminimierung und Ausschluss einiger phänomenologischer Fehler ermöglicht.
Die Frequenzmessung erfolgte durch Zählen der Perioden und Vergleich mit einer Atomuhr.\cite{Nieto2007} %``Schwingungen'' als Formulierung überprüfen -  Perioden
Für die Navigation wurde daraus direkt die aktuelle Flugbahn berechnet, wir wollen uns jedoch im Folgendem auf den – für
das Thema relevantere – 
Vergleich der gemessenen Geschwindigkeit mit der berechneten Geschwindigkeit beschränken. % Oder doch nicht?
Für die genauere Bestimmung der bahn muss man einige weitere Einflüsse berücksichtigen:

Da das Radiosignal zirkular polarisiert ist, muss bei der Berechnung die Rotation der Sonde berücksichtigt werden.
Dies führt pro Umdrehung der Sonde zu einer Phasenverschiebung von $(1+240/221) 2\pi$. %mehr, richtig?

Gerechnet wurde in der Standard-Epoche J200.0. %mehr

Die Bewegung der wurde in baryzentrischen Koordinaten gemäß ICRF beschrieben, % genauer
da die Erde jedoch sehr dynamisch ist muss – für eine hinreichend genaue Bestimmung der Sondenbahn –
die Bewegung der Antennen auf der Erdoberfläche unter der Berücksichtigung von Prezission, Nutation,
siderischen Rotation, Polarbewegung, der Gezeitenkräfte und Platentektonischen Bewegungen berücksichtigt werden.
Die Angaben zu Abbremsung, sowie zur Unregelmäßigkeit der Rotation, zur Polbewegung, die Love numbers\footnote{Von AEH
Love beschriebene ``Proportionalitätsfaktoren zwischen den verschiedenen Verzerrungen sowie dem sich einstellenden
Gravitationsfeld einer sphärisch symmetrischen, nichtrotierenden elastischen isotropen Kugel und einem äußerem an der
Kugel angreifenden Grafitationsgradienten''\cite{Dittus2006}} und der Chandler wobble\footnote{Spiralfomiges Schwingen
der Erdachse mit einer Periode von 435 Tagen} % Begriffe (genauer) erklären?
wurden dabei direkt aus Messungen mit Lunar Laser Ranging (LLR)\footnote{Beim LLR wird die Laufzeit von am Mond von
Spiegeln reflektierten Laserpulsen gemessen}, Satellite Laser Ranging (SLR)\footnote{Beim SLR wird die Laufzeit von
Laserpulsen zwischen einem Satellit und der Bodenstation gemessen} und Very Long Baseline Interferometry
(VLBI)\footnote{Dabei werden von zwei, interkontinental weit entfernten Radioteleskopen, die Signale inklusive
Zeitreferenz gespeichert und die Interferenz dieser Signale am Computer stimuliert} bestimmt.
Diese Daten wurden früher von Publikationen der International Earth Rotation Service (IERS) und der United States Naval Observatory (USNO) zusammengetragen. Heute werden die Daten vom ICRF bereitgestellt, zu welchen die Earth Orientation Parameters
(EOP) des JPL eine große Rolle beitragen.\cite{Anderson2002}

Aufgrund der Eigenrotation der Erde lässt sich andererseits aus der Modulation der  Doppler-Daten auch die
% – 3 dimensionale –
Position der Sonde berechnen. %Aber nich besonders gut, wie wir später sehen werden	 	% mehr

Die Beobachtungen der Bodenstationen wurden mit Zeitangaben nach der Universal Time 1 (UT1)\footnote{Durch astronomische
Beobachtung gewonnene und um Einflüsse der Polschwankungen (mit Perioden über 7 Tage) korrigierte, mittlere Ortszeit des
durch die Sternwarte von Greenwich führenden Nullmeridians} parametrisiert gespeichert. Für die Analysen und
Berechnungen musste diese Zeitangaben einerseits in die Internationale Atomzeit TAI, andererseits auch in die
Ephemeridenzeit umgerechnet werden. Auch hierfür verwendete man äußerst genaue Angaben zu Position, Geschwindigkeit und
Gravitationspotential der Antennen.\cite{Dittus2006}

% Es fehlt u.U. noch einige weitere Dinge die man hier erwähnen sollte.

