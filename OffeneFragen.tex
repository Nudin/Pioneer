
\subsection{Offene Fragen}

Obwohl die anomale Beschleunigung der Raumsonden Pioneer 10 und 11
bereits seit Anfang der 1980-er Jahre bekannt ist, gibt es noch viele
unbeantwortete Fragen\cite{Turyshev2010}. Sie und vor allem die fehlenden Antworten
k\"onnen wichtige Hinweise auf den wahren physikalischen Hintergrund
der Anomalie liefern.


\bigskip

Ein wichtiger Punkt ist, dass die exakte Richtung des
Beschleunigungsvektors immer noch unklar ist. Der Wert  $a_{p}=(8.74\pm
1.33)\times 10^{-8}\frac{\mathit{cm}}{s^{2}}$ wurde unter der Annahme
berechnet, dass die Beschleunigung in Richtung Sonne zeigt. Wegen
Unsicherheiten in der Doppler Navigation kann die Richtung der Anomalie
nur bis auf einen \"Offnungswinkel von 3{\textordmasculine} genau
bestimmt werden. Aufgrund der gro{\ss}en Entfernung der Sonden l\"asst
diese Ungenauigkeit insbesondere vier verschieden M\"oglichkeiten zu,
welche jeweils auf eine andere Ursache hinweisen.


\begin{figure}[htbn]
\begin{center}
\noindent    
\psfig{figure=images/richtung,width=\linewidth,height=\textheight,keepaspectratio}
\end{center}
\vskip -10pt
  \caption{Die möglichen Richtungen der Pioneer-Anomalie (1) Richtung Sonne (2) Richtung Erde (3) entlang des Geschwindigkeitsvektors (4) entlang der Spin-Achse\cite{Turyshev2010}}
\label{fig:flugbahn}
\end{figure} 

\bigskip

Falls die Sonden tats\"achlich in Richtung der Sonne beschleunigt werden
sollten, w\"are es ein Hinweis darauf, dass die Anomalie von einer
Kraft herr\"uhrt, die von dort Ausgeht. Da die Gravitation als Kraft in
Frage k\"ame, k\"onnte es ein Anzeichen daf\"ur sein, das eine
Modifikation das entsprechenden physikalischen Gesetzes notwendig
w\"are.


\bigskip

Die zweite M\"oglichkeit ist, dass die Beschleunigung in Richtung Erde
zeigt. Eine wahrscheinliche Ursache w\"aren hier die
Ausrichtungsman\"over der Sonden, wie zum Beispiel nach dem Vorbeiflug
an einem Planeten. Bei dieser Konstellation k\"onnte der Fehler aber
auch in der Hardware der DSN-Antennen oder der Signal\"ubertragung
liegen.


\bigskip

Es kommt noch in Frage, dass die Beschleunigung die Richtung des
Geschwindigkeitsvektors zeigt. Das k\"onnte darauf hindeuten, das die
Beschleunigung ihren Ursprung in einer Kraft hat, die von der
Tr\"agheit der Sonden ausgeht. Des Weiteren k\"onnten die Sonden bei
diesem Szenario durch Reibung, zum Beispiel durch Staub, abgebremst
werden.


\bigskip

Als Viertes bleibt noch die Richtung der Spin-Achse. Das w\"urde den
Versuch unterst\"utzen die Erkl\"arung f\"ur die Anomalie innerhalb der
Sonden zu finden. Korrekterweise sollte deshalb also von einer
Beschleunigung in Richtung des inneren Bereiches des Sonnensystems
gesprochen werden.


\bigskip

Das zu Stande kommen die Pioneer-Anomalie zum Beispiel durch
Hitzeabstrahlung der Sonden k\"onnte als Ursache komplett
ausgeschlossen werden, wenn man mit Sicherheit w\"usste, dass die
Beschleunigung \"uber einen sehr langen Zeitraum hin konstant ist.
\"Uber das Langzeitverhalten der Anomalie ist allerdings bis jetzt noch
zu wenig bekannt. Es ist, nach heutigem Kenntnisstand, durchaus
denkbar, dass die Anomalie nach einer gewissen Zeit wieder komplett
verschwindet, oder aber noch weiter w\"achst.


\bigskip

Diese \"Uberlegungen f\"uhren direkt zu den n\"achsten Fragen, deren
Beantwortung noch einiges Licht ins Dunkel bringen kann. In welchen
Distanzen genau kann diese Beschleunigung beobachtet werden? Die Daten
von Pioneer 10 und 11 best\"atigen ihre Existenz in einer Entfernung
von ungef\"ahr 20 - 70 AU. Doch was passiert au{\ss}erhalb dieses
Bereichs? Kann die Anomalie auch in der N\"ahe der Sonne beobachtet
werden? 

Die Flugbahnen beider Sonden lagen, um die Planeten zu besuchen, in der
Ekliptik. Wirkt sie sich auch auf Sonden aus, die sich senkrecht zur
Ekliptik bewegen?

