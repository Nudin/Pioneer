\subsection{Neue Analyse aller Vorhandener Daten}
Die Analyse von Anderson et al., 2002 betrachtete nur die Pioneer 10-Daten in der Zeit vom 3. Januar 1987 bis zum 22.
July 1998 (40 AU bis 70,5 AU) und die Pioneer 11-Daten vom 5. Januar 1987 bis zum 1. Oktober 1990 (22,4 AU bis 31,7
AU). Jedoch empfing man bis zum 27. April 2002 brauchbare Daten von Pioneer 10. (Die Daten von Pioneer 11 waren ab
Oktober 1990 nicht mehr brauchbar) Auch die anderen erwähnten Analysen bezogen sich immer auf etwa den selben Zeitraum.

Die Daten von Pioneer 10 aus dem Zeitraum von 1998 bis 2002 wurden also noch nicht untersucht. Noch
Erfolgversprechender wäre jedoch eine Analyse der frühen Daten. Die Anomalie wurde bereits ab 1979 beobachtet und durch
bessere Berechnung des solaren Strahlungsdruck könnten auch die noch früheren Daten wichtige Informationen liefern. Ees
ist daher geplant die kompletten Daten, vom Start der Sonden, bis zu den letzten verwertbaren Signalen neu zu
analysieren.

% Wohin mit ``bei 20 AU (~1980) sinkt der solare Strahlungsdruck auf unter 5*10^-8 cm/s^2''?

\bigskip
…
\bigskip

