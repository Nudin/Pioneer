\subsection{Neue Analyse aller Vorhandener Daten}
Die Pioneer Sonden waren über drei Jahrzehnte im Weltall. Die Analyse von Anderson et al., 2002 betrachtete nur die Daten aus etwa 11,5 Jahren von Pioneer 10 und 3,5 Jahren von Pioneer 11. (Genaue Daten in der Tabelle \ref{tab:andersondaten})
%Die Analyse von Anderson et al., 2002 betrachtete nur die Pioneer 10-Daten in der Zeit vom 3. Januar 1987 bis zum 22. July 1998 (40 AU bis 70,5 AU) und die Pioneer 11-Daten vom 5. Januar 1987 bis zum 1. Oktober 1990 (22,4 AU bis 31,7 AU).
Jedoch empfing man bis zum 27. April 2002 brauchbare Daten von Pioneer 10. (Die Daten von Pioneer 11 waren ab Oktober 1990 nicht mehr brauchbar) Auch die anderen erwähnten Analysen bezogen sich immer auf etwa den selben Zeitraum, teils sogar weniger.

\begin{table}[h]
\centering
\begin{tabular}{|c|c|c|}
\hline & Pioneer 10 & Pioneer 11 \\ 
\hline Zeitpunkt & 3.01.1987 - 22.07.1998  & 5.01.1987 - 1.10.1990 \\ 
\hline Entfernung & 40 - 70,5 AU & 22,4 - 31,7 AU \\ 
\hline 
\end{tabular}
\caption{Die von Anderson et al. verwendeten Daten}
\label{tab:andersondaten}
\end{table}

 
\begin{sidewaystable}[hn]
\newcommand{\mc}[3]{\multicolumn{#1}{#2}{#3}}
\begin{tabular}{|c|c|c|c|c|c|c|}
\hline  & \mc{3}{c|}{Pioneer 10} & \mc{3}{c|}{Pioneer 11} \\
\hline  & Zeitraum & Entfernung & Datenpunkte & Zeitraum & Entfernung & Datenpunkte \\
\hline Anderson et al, 1998 & 1987-1995 &  &  &  &  &  \\
\hline Anderson et al, 2002 & 3.01.1987 - 22.07.1998 & 40 - 70,5 AU & 19.403 & 5.01.1987 - 1.10.1990 & 40 - 70,5 AU & 10.252 \\
\hline … &   &   &   &   &   &   \\
\hline Verfügbar & -27.4.2002 &  &  & -10.1990 &  &  \\
\hline 
\end{tabular}
\caption{Übersicht über die Verwendeten Daten in allen Analysen}
\label{tab:andersondaten}
\end{sidewaystable}


Die Daten von Pioneer 10 aus dem Zeitraum von 1998 bis 2002 wurden also noch nicht untersucht. Noch
Erfolgversprechender wäre jedoch eine Analyse der frühen Daten. Die Anomalie wurde bereits ab 1979 beobachtet und durch
bessere Berechnung des solaren Strahlungsdruck könnten auch die noch früheren Daten wichtige Informationen liefern.

Es ist daher bereits seit längerem geplant die kompletten Daten, vom Start der Sonden, bis zu den letzten verwertbaren Signalen neu zu analysieren. Dies stellte sich jedoch als schwerer heraus als gedacht. Als Grund dafür werden veraltete Dateiformate, fehlende Daten und beschädigte Dateien genannt.\cite{Turyshev2010}

% Es ist geplant am ZARM in Zusammenarbeit mit dem JPL die gesammten Pioneer Daten vom Start bis zum letzten Signal neu
% zu analysieren	% aus Physikjournal
% die alten Daten wurden von Magnetbändern heruntergeladen und für neue Datenanalysen vorbereitet % aus Physikjournal


\bigskip
…
\bigskip

