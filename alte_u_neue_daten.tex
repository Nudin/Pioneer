\subsection{Neue Analyse aller Vorhandener Daten}\label{daten}
Die Pioneer Sonden waren über drei Jahrzehnte im Weltall. Die bisherigen Analysen betrachteten nur Daten aus etwa 11,5 Jahren von Pioneer 10 und 3,5 Jahren von Pioneer 11. Jedoch empfing man bis zum 27. April 2002 brauchbare Daten\footnote{Das letzte schwache Signal wurde am 23. Januar 2003 empfangen} von Pioneer 10. und immerhin bis zum Oktober 1990 brauchbare Daten von Pioneer 11.

Die erste Analyse im Jahr 1998 berücksichtigte nur die Daten von 1987 bis 1995 von Pioneer 10 und einen kürzeren Zeitfenster von Pioneer 11. Die ausführlichere Analyse 2002 betrachtete ein vergrößertes Zeitfenster. Die Daten der Sonde Pioneer 10 umfassten 19.403\footnote{Andere Quellen sprechen von 20.055 Datenpunkten\cite{Turyshev2004}} Datenpunkte in der Zeit vom 3. Januar 1987 bis zum 22. July 1998 auf Entfernungen von 40 AU bis 70,5 AU. Von Pioneer 11 betrachtete man 10.252\footnote{Andere Quellen sprechen von 10.616 Datenpunkte\cite{Turyshev2004}} Messpunkte vom 5. Januar 1987 bis zum 1. Oktober 1990 (22,4 AU bis 31,7 AU).% Quelle?
Dabei verwendete man Integrationszeiten zwischen 60 und 1.000 Sekunden.\cite{Turyshev2004}.

Auch die anderen erwähnten Analysen bezogen sich immer auf etwa den selben Zeitraum, teils sogar weniger. (siehe Tabelle \ref{tab:daten})

Dies lag einerseits daran, das diese neueren Daten bereits in passenden Dateiformaten vorlagen (siehe Kapitel \ref{messung}), und daran, dass in großer Entfernung der Einfluss einiger Effekte wie der Strahlungsdruck und Plasmaeffekte geringer ist, man diese also weniger genau berechnen musste.\cite{Nieto2005} Die späteren Daten wurden von Andersons Team schlicht und einfach nicht berücksichtigt, da sie zu Beginn der ersten Untersuchungen noch nicht vorlagen. %Quelle

Die Daten von Pioneer 10 aus dem Zeitraum von 1998 bis 2002 wurden also noch nicht untersucht.
Noch Erfolgversprechender wäre jedoch eine Analyse der frühen Daten. Die Anomalie wurde bereits ab 1979 beobachtet und durch bessere Berechnung des solaren Strahlungsdruck könnten auch die noch früheren Daten wichtige Informationen liefern.
Ein kompletter Datensatz ab dem Start der Sonden, bis zu den letzten brauchbaren Signalen würde etwa 80.000 Datenpunkte von 30 Jahren Kommunikation mit Pioneer 10  über Distanzen von ungefähr 1 AU bis 80 AU sowie 50.000 Datenpunkte aus 17.5 Jahren (1 bis 31,7 AU) von Pioneer 11 umfassen.\cite{Turyshev2004} Es gibt also etwas 17,5 Jahre Pioneer 10 und 12,5 Jahre Pioneer 11 Daten die bisher nicht ausgewertet wurden. Weniger als ein Viertel der Daten wurden bisher genutzt. Auch wenn lange nicht alle Datenpunkte gleich brauchbar sind, würde eine erneute Analyse viele weitere Erkenntnisse bringen.

\begin{table}[ht]
\centering
\begin{tabular}{|c|c|c|}
\hline & Pioneer 10 & Pioneer 11 \\ 
\hline Zeitpunkt & 3.01.1987 - 22.07.1998  & 5.01.1987 - 1.10.1990 \\ 
\hline Entfernung & 40 - 70,5 AU & 22,4 - 31,7 AU \\ 
\hline 
\end{tabular}
\caption{Die von Anderson et al. verwendeten Daten}
\label{tab:andersondaten}
\end{table}


Es ist daher bereits seit längerem geplant die kompletten Daten, vom Start der Sonden, bis zu den letzten verwertbaren Signalen neu zu analysieren. Dafür müsste man die kompletten Daten mit einheitlichen Verfahren neu aufbereiten und auswerten. Dies stellte sich jedoch als schwerer heraus als gedacht. In den über 30 Jahren haben sich die Dateiformate, die Navigationssoft und -hardware, das DSN und die beteiligten Personen mehrfach geändert. Mann hat unter anderem die folgenden Probleme überwinden müssen:\cite{Turyshev2010}
\begin{itemize}
\item Die Daten waren auf verschiedene Archive verteilt, einige existierten sogar nur noch bei (ehemaligen) Mitarbeitern Zuhause.
\item Die Dateien waren in unterschiedlichen Formaten (abweichen von den oben beschriebenen), einige davon waren veraltet, schlecht dokumentiert und nicht problemlos in heutige Formate umzuwandeln.
\item Dateien waren oft unvollständig, es fehlen teils kritische Informationen.
\item Die Dateien wurden unterschiedlich behandelt, so wurden bei Dateien manchen die Dopplerdaten um die Effekte des Raumsonden Spins korrigiert, bei andern nicht. (siehe Kapitel \ref{messung})
\item Einige Dateien oder Magnetbänder waren beschädigt.\footnote{So konnte Marward beispielsweise in seiner Arbeit die Daten vom Juni 1990 bis zum Juni 1991 nicht verwenden, da das entsprechende Magnetband im NSSDC nicht mehr lesbar war.\cite{Markwardt2002}} In einigen Fällen konnten die Dateien repariert werden, in andern waren die Dateien irreparabel beschädigt und die Daten waren verloren.
\end{itemize}
Trotzdem hat man es bis zum November 2009 geschafft, die Daten so weit wie möglich vollständig zusammen zutragen und herzurichten.
Die ursprünglichen Datenquellen sind: % Übersetzung? ("initially", turysehv2010 - S.56)
\begin{itemize}
\item Die archivierten Dateien am JPL und dem Deep Spache Network
\item Die archivierten Dateien am National Space Science Data Center (NSSDC)
\item Daten welche von einzelnen Mitarbeitern des JPL archiviert wurden.
\end{itemize}
Die Pioneer 10-Daten von vor 1980 haben sich dabei als weitgehend unbrauchbar herausgestellt, die Daten danach sind jedoch durchgehend brauchbar. Die Pioneer 11-Daten sind von Mitte 1978 bis zum Ende brauchbar. Die ausführliche Analyse dieses umfangreichen Datensatzes ist derzeit in Arbeit.\cite{Turyshev2010} Mit den Ergebnissen dieser Analyse kann in Kürze gerechnet werden. % wirklich?

\begin{figure}[htnb]
\begin{center}
\noindent    
\psfig{figure=images/cOMBINED,width=\linewidth,height=\textheight,keepaspectratio}
\end{center}
\vskip -10pt
  \caption{
Die obere Kurze zeigt den verlauf des Strahlungsdrucks an, die untere die der Anomalie. In der Mitte die Kombination von beiden. Die Grafik erschien erstmals in \cite{Anderson1992}}
\label{fig:forces}
\end{figure} 

\begin{figure}[htnb]
\begin{center}
\noindent    
\psfig{figure=images/correlate,width=\linewidth,height=\textheight,keepaspectratio}
\end{center}
\vskip -10pt
  \caption{
Verlauf der Anomalen Beschleunigung in Abhängigkeit von der Entfernung in Astronomischen Einheiten. Diese Tabelle zeigt das Resultat einer groben Auswertung der Daten an, sie ist noch kein Ergebnis der in Arbeit befindlichen, genauen Analyse der Daten über den gesamten Zeitraum. Die Ersterscheinung dieser Grafik konnte nicht ermittelt werden, sie taucht jedoch in zahlreichen Arbeiten auf, darunter\cite{Anderson2002}\cite{Nieto2005}
}
\label{fig:anomalie}
\end{figure} 

Das wir die schwerer zu analysierenden frühen Daten erst jetzt im Detail untersuchen, bringt uns den Vorteil, dass wir aus den bisherigen Auswertungen Lektionen darüber gelernt haben, wie man mit den Daten richtig umgeht und wie man den Einfluss der diversen Effekte und der Manöver berücksichtigt.\cite{Nieto2005}
Außerdem haben sich inzwischen die Ephemerieden, sowie die Modelle zu Beschreibung der Position der DSN-Antennen erheblich verbessert. Dies ist durch Fortschritte in der Technik – insbesondere bei GPS- und VLBI-gestützter Technologie – neue Erkenntnisse zum Dopplerortung von Raumsonden und neuen Datenverarbeitungsalgorithmen ermöglicht.
Mann wird diese verbesserten Erdmodelle vom IERS übernehmen und die neuesten Ephemerieden verwenden. Dadurch wird es möglich sein die Genauigkeit der Position der DSN-Antennen um zwei Größenordnungen auf 1 cm zu verbessern und so nicht nur die konstante Anomalie genauer zu beschreiben, sondern auch die zeitlich periodischen Terme zu überprüfen.\cite{Turyshev2004}

Durch die deutliche Vergrößerung der Anzahl an verfügbaren Daten ist damit zu rechnen, dass man die Anomalie überprüfen und genauer sowie zuverlässiger bestimmen kann.
Ermutigend ist hier, dass gezeigt wurde, dass die letzten Datenpunkte, der früheren, noch nicht ausgewerteten, Daten – nach der Entfernung der jährlichen Schwankungen – mit den Untersuchungen von \cite{Anderson2002} statistisch konsistent sind.\cite{Nieto2005}

Wichtiger ist jedoch das man bei einer Analyse über einen so langen Zeitraum den Ursprung der Anomalie in der Thermischen Abstrahlung wesentlich sicherer ausschließen können müsste. Da der Radioisotopengenerator mit einer Halbwertszeit von 87,7 Jahren immer weniger Wärme abgibt, müsste über einen längeren Zeitraum hinweg die Anomalie sinken. Desto länger der betrachtete Zeitraum ist, desto sicherer müsste man einen zeitliche Veränderung feststellen.
Im ursprünglich betrachteten Zeitraum von 11,5 Jahren wäre die Energie des RTGs nur um etwa 9\% gefallen, im ganzen beobachtbaren Zeitraum vom 22 Jahren liegt die abnahmen jedoch bereits bei 16\% was sich deutlich nachweisen lassen müsste.

Von besonderen Interesse könnten auch die frühen Daten sein:
Die bisherigen Analysen deuten darauf hin, dass die Anomalie erst ab einer Entfernung von etwa 15 AU auftritt. Es ist erfolgversprechend den Zeitpunkt bei welchem die Anomalie anfängt aufzutreten zu analysieren. Die Grafik \ref{fig:anomalie} zeigt beim zweiten Punkt einen auffallend großen Fehler, sollte es sich dabei um einen korrekten, wenn auch groben, Messwert handeln und nicht um ein Problem  mit dem Verhältnis zwischen Signal und Rauschen, so könnte dies ein deutlicher Hinweis auf ein Beginnen der Anomalie sein. % "turn-on"
Die große Messunsicherheit würde dadurch zustande kommen, da die Punkte in der Grafik \ref{fig:anomalie} Durchschnittswerte über Integrationszeiten von etlichen Monaten bis hin zu einem Jahr angeben. Der zweite Pioneer 11-Datenpunkt beginnt also schon for der Kreuzung der Saturnbahn, endet jedoch erst danach.
Die Analyse dieser Zeit könnten eine wichtige Rolle im Bezug auf die Theorie dunkle Materie könnte die Anomalie verursachen, sein. Auch Die MOND-Theorie müsste hier Anzeichen hinterlassen: die Anomalie müsste dafür beim Vorbeiflug am Saturn von Pioneer 11 beziehungsweise beim Vorbeiflug von Pioneer 10 am Jupiter erstmals auftreten.\cite{Nieto2005} %genauer (in Nieto2005)


% Es ist geplant am ZARM in Zusammenarbeit mit dem JPL die gesammten Pioneer Daten vom Start bis zum letzten Signal neu
% zu analysieren	% aus Physikjournal
% die alten Daten wurden von Magnetbändern heruntergeladen und für neue Datenanalysen vorbereitet % aus Physikjournal

\FloatBarrier

\begin{sidewaystable}[hnt]
\newcommand{\mc}[3]{\multicolumn{#1}{#2}{#3}}
\begin{tabular}{|c|c|c|c|c|c|c|}
\hline  & \mc{3}{c|}{Pioneer 10} & \mc{3}{c|}{Pioneer 11} \\
\hline  & Zeitraum & Entfernung & Datenpunkte & Zeitraum & Entfernung & Datenpunkte \\
\hline Anderson et al, 1998 & 1987-1995 &  &  &  &  &  \\
\hline Anderson et al, 2002 & 3.01.1987 - 22.07.1998 & 40 - 70,5 AU & 19.403 & 5.01.1987 - 1.10.1990 & 40 - 70,5 AU & 10.252 \\
\hline Markward & 1987-1994 & 40-69 AU & &\mc{3}{c|}{-} \\
\hline … &   &   &   &   &   &   \\
\hline Verfügbar & -27.4.2002 &  &  & -10.1990 &  &  \\
\hline 
\end{tabular}
\caption{Übersicht über die Verwendeten Daten in allen Analysen}
\label{tab:daten}
\end{sidewaystable}