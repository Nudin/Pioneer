\section{ Dunkle Materie}\label{Dm}

Nachdem die Pioneer-Anomalie entdeckt wurde, kamen erste Theorien auf,
dass die Anomalie ein Indiz f\"ur Dunkle Materie in unserem
Sonnensystem sein k\"onnte. Verschiedene Verteilungen der Dunklen
Materie wurden durchgerechnet und so kann zum Beispiel eine Scheibe mit
einer Dichte von etwa $4\cdot 10^{-16}\mathit{kg}/m^{3}$ im
\"au{\ss}eren Sonnensystem die Anomalie erkl\"aren. Allerdings d\"urfte
diese Dunkle Materie nicht wie leuchtende Materie gravitativ
beeinflussbar sein\cite{Turyshev2010}. Sprich, diese Scheibe wird nicht
von Planeten beeinflusst und somit nicht angeh\"auft. Dies ist jedoch
unwahrscheinlich. Wenn die Pioneer-Anomalie einen gravitativen Ursprung
h\"atte, so m\"usste sie nach Newton mit $1/r^{2}$ abnehmen. Eine
derartige Abnahme h\"atte man in den 2002 vorliegenden Datenintervallen
beobachten m\"ussen, was aber nicht geschehen ist. Ein weiteres
Argument gegen die Dunkle Materie als Grund f\"ur die Pioneer-Anomalie
ist die Masse und Dichte von Dunkler Materie, die n\"otig w\"aren um
den Effekt zu verursachen. Es w\"are in einem Abstand von 50 AU eine
Masse von mindestens $3\cdot 10^{-4}$ Sonnenmassen (sprich
$5,967\cdot 10^{26}\mathit{kg}$) mit einer Dichte von $6,0\cdot
10^{18}\mathit{kg}/\mathit{AU}^{3}$ n\"otig. Xu et al. \cite{Xu2008} argumentieren jedoch, dass \"uber die gesamte Lebensdauer
von unserem Sonnensystem von $4,5\cdot 10^{9}a$ sich maximal eine Dichte von
$2,0\cdot 10^{17}\mathit{kg}/\mathit{AU}^{3}$ h\"atte ansammeln
k\"onnen. Au{\ss}erdem sei die Masse an Dunkler Materie im gesamten
Sonnensystem nur ca. {}10\textsuperscript{20 }kg, womit die
Pioneer-Anomalie in keiner Weise erkl\"art w\"are.


