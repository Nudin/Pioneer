
\subsection{Variabler Teil}
Während in Normalfall die Anomalie als konstante Beschleunigung angesehen wird, wiesen betreits Anderson et al. in
ihrer Arbeit im Jahr 2002 darauf hin, dass es periodische Anteile, von etwa 10\%, in der Pioneeranomalie zu
geben scheint.
Im Jahr 2008 zeigte die „Groupe Anomalie Pioneer“ (GAP) – ein Zusammenschluss von etlichen französischen Forschungseinrichtungen –
das sich die Qualität des Fits nennenswert steigern lässt, wenn man zusätzlich zur konstanten Beschleunigung
periodische Terme verwendet.
Dabei ist es ihnen gelungen eine Beziehung zwischen dem Unterschied der Azimutalwinkel zwischen Sonde und Erde, sowie
den zeitlich veränderlichen Anteilen der Pioneeranomalie zu finden. %bäh

Die periodischen Anteile des Signals lassen sich sehr gut auf der auf Abb. FIXME dargestellten Spektralanalyse
erkennen. Die Drei großen Peaks liegen bei $f_1=0.9974\pm0.004\ Tagen$, $f_2=\frac12(0.9974\pm0.004)\ Tagen$ und 
$f_3=189\pm32\ Tagen$. Bedenkt man das 1.0 siderischer Tag = 0.9972 Tage ist, so entspricht dies genau
halbtägigen, täglichen und halbjährlichen Schwankungen.
Anderson et al. ziehen dafür Messfehler – wie Fehler in den Ephemeriden oder der Ausrichtung der Drehachse der Erde
oder fehlerhafte Koordinaten der Messstationen – in Betracht. % DOPPELT, s.u.
Die Gruppe um Levey (GAP) hält dies jedoch für unwahrscheinlich, da diese Daten durch andere Beobachtungsmethoden
gestützt werden. % wo steht das?
Sie nehmen in ihrer Untersuchung an, dass durch eine beliebige Ursache die Ausbreitung des Tracking-Signals auf dem Weg
zwischen Raumsonde und Erde verändert wird. Sie beschreiben die Ursache als Funktion des Azimutalwinkel zwischen Sonde und Erde und fitten
dann die Fouriekoeffizienten. Dieses geometrische Modell beschreibt sowohl die täglichen, also auch die jährlichen
Schwankungen und verringert die Abweichung von den Messwerten von 9.8 mHz auf 5.5 mHz. Auch die Spektralanalyse (
Abb. FIXME) dieses fits zeigt die Verbesserung deutlich.\cite{Levy2008} % Besser zitieren
%mehr, weiter, ...

Anderson et al. führen diese Schwankungen darauf zurück, dass bei den Berechnungen die Daten zur Position und Bewegung
der Bodenstationen nicht genau genug waren.\cite{Dittus2006} %Ordentliche, genaue Quelle suchen, DOPPELT, s.o.
