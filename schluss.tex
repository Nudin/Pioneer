\section{Diskussion}

Im Verlauf des Studiums der Arbeiten zur Pioneer-Anomalie sind uns viele unterschiedliche Erklärungsmodelle begegnet. Einige haben wir hier in unserer Arbeit vorgestellt. Unabhängig davon, welche Ursache die Anomalie hat, wird das Ergebnis ein großer Informationsgewinn für die Raumfahrt sein. Wenn MOND oder Dunkle Materie für die Anomalie verantwortlich wären, so muss man diese Phänomene intensiv erforschen und in der Berechnung zukünftiger Missionen berücksichtigen. Andernfalls wären die Pioneer-Missionen eine sehr genaue Überprüfung der bekannten klassischen Gravitationsgesetze darstellen. Sollte ein sondeninterner Fehler als Quelle der Anomalie ausgemacht werden, so hat dies Auswirkungen auf die Konstruktion künftiger Raumsonden. Ein möglicher Rechenfehler kann als Ursache mittlerweile ausgeschlossen werden. Die Berechnungen wurden wie in \ref{berechnungen} gezeigt durch viele unabhängige Programme, die teilweise speziell zur Auswertung der Anomalie geschrieben wurden, durchgeführt. Alle Programme haben wiederholt den selben Wert errechnet, womit ein Rechenfehler als Ursache der Pioneer-Anomalie mit großer Sicherheit ausgeschlossen werden kann.

\bigskip
\subsection{Wertung der unterschiedlichen Erklärungsmodelle}
Die MOND-Theorie (Kap. \ref {MOND}) und die Theorie der Dunklen Mateire (Kap. \ref {Dm}) sind unserer Meinung nach nicht die Quellen für die anormale Beschleunigung von Pioneer 10 und 11. Wäre die Anomalie gravitativen Ursprungs, hätte man messbare Auswirkungen auf die Bahnen der äußeren Planeten. Diese Bahnen wurden über einen Zeitraum von mehreren Jahrhunderten sehr genau beobachtet und berechnet. In der Beobachtungen wurden keine Abweichungen zu den Berechnungen festgestellt. Deswegen bezweifeln wir stark, dass die Pioneer-Anomalie in Zusammenhang mit MOND oder Dunkler Materie steht.
\bigskip
Dunkle Energie (Kap. \ref{De}) kommt ebenfalls nicht in Frage die Anomalie zu erklären. Obwohl die Hubblebeschleunigung $cH_0$ zwar dem Betrag von $a_p$ gefährlich nahe kommt, stimmt jedoch das Vorzeichen nicht. Die Auswirkungen der Ausdehnung des Universums sind auf den Längenskalen der Pioneer-Sonden kaum messbar.
\bigskip
Als vielversprechende Erklärung hat sich die unterschätzte Hitzeabstrahlung der Sonden herauskristalisiert. Wie in Kap. \ref{Hitze} angesprochen wurde die thermische Abstrahlung und auch ihre Reflexion an der Sonde in der ursprünglichen Arbeit von Anderson et. al \cite{Anderson2002} stark unterschätzt. Da alle anderen internen Fehlerquellen relativ genau modelliert wurden und externe Fehlerquellen als Ursache ausgeschlossen wurden, nehmen wir an, dass $a_p$ durch die thermische Emission zustande kommt. Die Untersuchung dieser Ursache ist zur Zeit bestandteil einiger Studien. Ende März 2011 wurde eine Arbeit von portugisischen Wissenschaftlern\cite{port2011} veröffentlicht, die eine Erklärung für die Anomalie aufgrund der thermischen Abstrahlung gefunden haben. Die zentrale Annahme in dieser Arbeit ist die, dass nicht nur die RTGs als Lambertstrahler angenommen werden, sondern auch das Fach für die technischen Geräte als Lambertstrahler modelliert wird. Außerdem wird die Reflexion von thermischer Emission an der Sonde mit in die Berechnungen mit einbezogen. Diese Arbeit wird zur Zeit vom JPL, namentlich von Anderson, Nieto und Turyshev überprüft. Wir sind zuversichtlich, dass damit die Pioneer-Anomalie gelöst ist.



\section{Was wir schon immer an das Ende einer Hausarbeit schreiben wollten}

\bigskip
Dunkle Energie ist doof.

\bigskip
Atomkraft-Nein Danke!

\bigskip
Pluto ist ein Planet (Michael denkt das nicht)

\bigskip
Beteigeuze schrumpft!
\bigskip

Placebo-Bänder für ALLE!
