\section{Diskussion}

Im Verlauf des Studiums der Arbeiten zur Pioneer-Anomalie sind uns viele unterschiedliche Erklärungsmodelle begegnet. Einige haben wir hier in unserer Arbeit vorgestellt. Unabhängig davon, welche Ursache die Anomalie hat, wird das Ergebnis ein großer Informationsgewinn für die Raumfahrt sein. Wenn MOND oder Dunkle Materie für die Anomalie verantwortlich wären, so muss man diese Phänomene intensiv erforschen und in der Berechnung zukünftiger Missionen berücksichtigen. Andernfalls würde die Anomalie eine sehr genaue Überprüfung der bekannten klassischen Gravitationsgesetze darstellen. Sollte ein sondeninterner Fehler als Quelle der Anomalie ausgemacht werden, so hat dies Auswirkungen auf die Konstruktion künftiger Raumsonden. Die Berechnungen wurden wie in \ref{berechnungen} gezeigt durch viele unabhängige Programme, die teilweise speziell zur Auswertung der Anomalie geschrieben wurden, durchgeführt. Alle Programme haben wiederholt den selben Wert errechnet, womit ein Rechenfehler oder Softwarefehler als Ursache der Pioneer-Anomalie mit großer Sicherheit ausgeschlossen werden kann.

\bigskip

\subsection{Wertung der unterschiedlichen Erklärungsmodelle}
Die MOND-Theorie (Kap. \ref{MOND}) und die Theorie der Dunklen Mateire (Kap. \ref{Dm}) sind unserer Meinung nach nicht die Quellen für die anormale Beschleunigung von Pioneer 10 und 11. Wäre die Anomalie gravitativen Ursprungs, hätte dies messbare Auswirkungen auf die Bahnen der äußeren Planeten. Diese Bahnen wurden über einen Zeitraum von mehreren Jahrhunderten sehr genau beobachtet und berechnet. In der Beobachtungen wurden keine Abweichungen zu den Berechnungen festgestellt. Deswegen bezweifeln wir stark, dass die Pioneer-Anomalie in Zusammenhang mit MOND oder Dunkler Materie steht.

\bigskip

Dunkle Energie (Kap. \ref{De}) kommt ebenfalls nicht in Frage die Anomalie zu erklären. Obwohl die Hubblebeschleunigung $cH_0$ zwar dem Betrag von $a_p$ recht nahe kommt, stimmt jedoch das Vorzeichen nicht und die Ausdehnung des Universums hat auf die Längenskalen, in denen sich die Pioneer-Sonden bewegen, keine messbaren Auswirkungen.

\bigskip

Als vielversprechende Erklärung hat sich die unterschätzte Hitzeabstrahlung der Sonden herauskristalisiert. Wie in Kap. \ref{Hitze} angesprochen wurde die thermische Abstrahlung und auch ihre Reflexion an der Sonde in der ursprünglichen Arbeit von Anderson et. al \cite{Anderson2002} stark unterschätzt. Da alle anderen internen Fehlerquellen relativ genau modelliert wurden und externe Fehlerquellen als Ursache ausgeschlossen wurden, nehmen wir an, dass $a_p$ durch die thermische Emission zustande kommt. Dies würde allerdings zur Folge haben, dass $a_p$ nicht konstant ist, sondern, wie die Halbwertszeit von Pu-238 in den RTGs, abnimmt. Wie in Kap. \ref{daten} angesprochen konnte dieser Effekt aufgrund des begrenzten Datensatzes noch nicht überprüft werden. Die laufenden Analysen der gesamten Daten könnte darüber Aufschluss geben. Die Untersuchung der thermischen Emission ist zur Zeit Bestandteil einiger Studien. Ende März 2011 wurde auf dem Preprint-Server ArXiv.org eine Arbeit von portugisischen Wissenschaftlern veröffentlicht\cite{port2011}, die eine Erklärung für die Anomalie aufgrund der thermischen Abstrahlung gefunden haben wollen – eine Veröffentlichung in in Physical Review D wird angestrebt. Die zentrale Annahme in dieser Arbeit ist die, dass nicht nur die RTGs als Lambertstrahler angenommen werden, sondern auch das Fach für die technischen Geräte als Lambertstrahler modelliert werden muss. Außerdem wird die Reflexion von thermischer Emission an der Sonde mit in die Berechnungen mit einbezogen. Am 20. April 2011 (nach Abgabe dieser Arbeit) veröffentlichen Benny Rievers und Claus Lämmerzahl ebenfals auf ArXiv.org eine Arbeit welche die Anomalie ebenfalls durch thermische Abstrahlung erklärt.\cite{Rievers2011}\footnote{Sie betonen jedoch ausdrücklich das dies nicht die Fly-by-Anomalie erklärt.} Sie simulieren die Thermische Abstrahlung jedoch mit der Methode der finiten Elemente. Somit hat man eine unabhängige Überprüfung. Die enge zeitliche Abfolge der beiden Veröffentlichungen, sowie Andeutungen von Claus Lämmerzahl uns gegenüber lassen darauf schließen, dass es hier offensichtlich ein Wettrennen zwischen den beiden Gruppe gab.
Diese Erklärungsmodelle werden zur Zeit vom JPL, namentlich von Anderson, Nieto und Turyshev überprüft. Wir sind zuversichtlich, dass damit die Pioneer-Anomalie gelöst ist.
